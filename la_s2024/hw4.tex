\documentclass[12pt]{article}

\usepackage[margin=1.25in]{geometry}

\usepackage{graphicx}
\usepackage{amsmath,amsthm,amssymb}



\theoremstyle{definition}
\newtheorem{problem}{Problem}
\newenvironment{solution}
  {\textbf{Solution.} }
  {}


\usepackage{enumitem}
\setlist[itemize]{itemsep=-.15em, topsep=-.5em}
\setlist[enumerate]{itemsep=-.15em, topsep=-.5em}
\setlist[enumerate,1]{label={(\alph*)}}

\usepackage{hyperref}

\setlength{\parskip}{1em}
\setlength{\parindent}{0em}

\renewcommand{\d}{\mathrm{d}}
\newcommand{\T}{\mathsf{T}}
\newcommand{\range}{\operatorname{range}}
\renewcommand{\null}{\operatorname{null}}


\usepackage[]{xcolor}

\begin{document}

\textbf{\Large\sffamily{Homework 4\hfill Linear Algebra I}}
    
\vspace{-1.8em}
\hrulefill


\textbf{Instructions:}
    \begin{itemize}
        \item Due March 26 at 11:59pm on \href{https://www.gradescope.com/courses/709136}{Gradescope}.
        \item You must follow the submission policy in the \href{https://courses.chen.pw/la_s2024/syllabus.html}{syllabus} 
\end{itemize}
   
\vspace{.5em}


\begin{problem}
As mentioned in lecture, we can express abstract linear operatons on abstract linear spaces in terms of matrices and vectors. 
Here we will work through some of the math that would be required to manipulate polynomials on a computer.\footnote{This is what makes stuff like numpy's \href{https://numpy.org/doc/stable/reference/routines.polynomials.html}{\texttt{polynomial}} module work.}

For conveneience, we will restrict to polynomials of degree at most 4.

We are used to working with the monomial basis:
\[
m_0(x) = 1, \quad
m_1(x) = x, \quad
m_2(x) = x^2, \quad
m_3(x) = x^3, \quad
m_4(x) = x^4
\]
For certain numerical reasons, we might also be interested in the Chebyshev Basis:
\[
T_0(x) = 1, \quad
T_1(x) = x, \quad
T_2(x) = 2x^2 - 1, \quad
T_3(x) = 4x^3 - 3x, \quad
T_4(x) = 8x^4 - 8x^2 + 1    
\]

\begin{enumerate}
    \item Write down the differentiation matrices $D_m\in \mathcal{L}(\mathcal{P}_4,\mathcal{P}_4)$ the integration matrix (for a definite integral over $[a,b]$) $T_m \in \mathcal{L}(\mathcal{P}_4,\mathbb{R}^1)$ (with respect to the monomial basis for the input and output). The integration matrix should depend on $a$ and $b$.
    \item Repeat (a) for the Chebyshev basis for the input and output spaces (call these matrices $D_c$ and $D_m$.
    \item Write a the matrix $M$ which converts a Chebyshev polynomial to a monomial polynomial, and the matrix $N$ which converts a monomial polynomial to a Chebyshev polynomial. 
\end{enumerate}

    
\end{problem}

\begin{problem}
    Let $D_m$, $D_c$, $M$ and $N$ be the matrices defined in the previous problem.
    \begin{enumerate}
        \item Describe in words what the following matrices are doing as operations:
        \begin{itemize}
            \item $D_m M$
            \item $D_c N$
            \item $M N$
            \item $T_m D_m D_m$
        \end{itemize}
        \item verify by hand (meaning multiply out the matrices) the following:
        \begin{itemize}
            \item $M N = I$
            \item $N M = I$
            \item $D_m M = D_c$
            \item $D_c N = D_c$
            \item $(D_m)^5 = 0$.
        \end{itemize}
    \end{enumerate}

\end{problem}

\begin{problem}
    Implement the above operations in a computer program. 
    I have provided started python code here: \url{https://colab.research.google.com/drive/1qqrpd-a7NnAo7uooQs1CVgnbnwgpIvS6}.

    Include the functions you implement in your submission.
    If you are using latex, you can use the \texttt{listings} package.
\end{problem}


\begin{problem}~
\begin{enumerate}
    \item Suppose $A$ is a $m\times n$ matrix and $B$ is a $n\times p$ matrix. Prove $(AB)_{j,:} = (A)_{j,:}B$ for each $j=1, \ldots, m$.
    \item Prove that matrix multiplication is associative: $(AB)C = A(BC)$ for any matrices $A,B,C$ for which the products are defined.
    \item Suppose $v_1,\ldots,v_n$ is a basis for $V$ and $w_1, \ldots, w_m$ is a basis for $W$. Show that if $S,T\in\mathcal{L}(V,W)$, then $\mathcal{M}(S+T) = \mathcal{M}(S) + \mathcal{M}(T)$.
    \item Prove that $(AC)^T = C^TA^T$ for any matrices $A$ and $C$ for which $(AC)$ and $A^TC^T$ are defined.
    \item Suppose $V$ is finite dimensional and $\dim(V) >1$. Prove the set of noninvertible linear maps from $V$ to itself is not a subspace of $\mathcal{L}(V,V)$.
    \item Suppose $V$ is finite dimensional and $S,T\in\mathcal{L}(V,V)$. Prove that $ST$ is invertible if and only if $S$ and $T$ are invertible.
    \item Suppose $W$ is finite dimensional and $S,T \in \mathcal{L}(V,W)$. Prove that $\range(S) = \range(T)$ if and only if there exists an invertible $E\in\mathcal{L}(V,V)$ such that $S = T E$.
\end{enumerate}
\end{problem}

\end{document}

\documentclass[12pt]{article}

\usepackage[margin=1.25in]{geometry}

\usepackage{graphicx}
\usepackage{amsmath,amsthm,amssymb}

% you will probably have to remove the font-setup
\usepackage[no-math]{fontspec}
\setmainfont[
    BoldFont = Vollkorn Bold,
    ItalicFont = Vollkorn Italic,
    BoldItalicFont={Vollkorn Bold Italic},
    RawFeature=+lnum,
]{Vollkorn}

\setsansfont[
    BoldFont = Lato Bold,
    FontFace={l}{n}{*-Light},
    FontFace={l}{it}{*-Light Italic},
]{Lato}

\usepackage{unicode-math}
\mathitalicsmode=1

\setmathfont[
mathit = sym,
mathup = sym,
mathbf = sym,
math-style = TeX, 
bold-style = TeX
]{Wholegrain Math}
\everydisplay{\Umathoperatorsize\displaystyle=4ex}
\AtBeginDocument{\renewcommand\setminus{\smallsetminus}}
% font setup ends here


\theoremstyle{definition}
\newtheorem{problem}{Problem}
\newenvironment{solution}
  {\textbf{Solution.} }
  {}


\usepackage{enumitem}
\setlist[itemize]{itemsep=-.15em, topsep=-.5em}
\setlist[enumerate]{itemsep=-.15em, topsep=-.5em}
\setlist[enumerate,1]{label={(\alph*)}}

\usepackage{hyperref}

\setlength{\parskip}{1em}
\setlength{\parindent}{0em}

\renewcommand{\d}{\mathrm{d}}
\newcommand{\T}{\mathsf{T}}

\usepackage[]{xcolor}

\begin{document}

\textbf{\Large\sffamily{Homework 2\hfill Linear Algbera I}}
    
\vspace{-1.8em}
\hrulefill

\textbf{Instructions:}
    \begin{itemize}
        \item Due Feb 19 at 11:59pm on \href{https://www.gradescope.com/courses/709136}{Gradescope}.
        \item You must follow the submission policy in the \href{https://courses.chen.pw/la_s2024/syllabus.html}{syllabus} 
\end{itemize}
   
\vspace{.5em}

\begin{problem}[Span]~
\begin{enumerate}
\item Let $\vec{v}_1, \ldots, \vec{v}_k \in V$. Prove $\operatorname{span}(\vec{v}_1, \ldots, \vec{v}_k)$ is a subspace of $V$.

\item Show that $\operatorname{span}(\vec{v}_1, \ldots, \vec{v}_k)$ is the smallest subspace of $V$ containing $\vec{v}_1, \ldots, \vec{v}_k$.

\item Find a list of four distinct vectors in $\mathbb{R}^3$ whose span equals 
\[ \{ (x,y,z) \in \mathbb{R}^3 : x+y+z = 0\}. \]

\item Prove or give a counterexample: If $\vec{v}_1, \vec{v}_2, \ldots, \vec{v}_m$ is a linearly independent list of vectors in $V$ and $\lambda \in \mathbf{F}$ with $\lambda \neq 0$, then $\lambda \vec{v}_1, \lambda \vec{v}_2, \ldots, \lambda \vec{v}_m$ is linearly independent.
\end{enumerate}
\end{problem}

\begin{problem}[Independence]~
\begin{enumerate}
    \item Let $\vec{v}_1, \ldots, \vec{v}_k \in V$. Prove that if $\vec{v}_1, \ldots, \vec{v}_k$ are linearly dependent if and only if there exists $i \in \{1, \ldots, k\}$ such that $\vec{v}_i \in \operatorname{span}(\vec{v}_1, \ldots, \vec{v}_{i-1}, \vec{v}_{i+1}, \ldots, \vec{v}_k)$.
    
    \item Show that if the set of vectors $\vec{v}_1, \ldots, \vec{v}_k$ are linearly independent, then none of the $\vec{v}_i$ are the zero vector.
    
    \item Prove or give a counterexample: If $\vec{v}_1, \ldots, \vec{v}_m$ and $\vec{w}_1, \ldots, \vec{w}_m$ are linearly independent lists of vectors in $V$, then the list $\vec{v}_1+\vec{w}_1, \ldots, \vec{v}_m+\vec{w}_m$ is linearly independent.
    \end{enumerate}
\end{problem}

\begin{problem}[Basis]~
\begin{enumerate}
    \item Show that a linear space spanned by a finite set of vectors has a basis. (This is Lemma 2 in Lax, but the proof is a bit too terse, so you should fill in the details. In particular, he writes ``So we can drop that $x_i$'' but does not explain what should happen after we drop $x_i$, and he writes that we can repeat this until they are linearly independent but does not justify why it will eventually become independent.)
    \item Show every vector space spanned by a finite set of vectors has a basis.
    \item Suppose $\vec{v}_1, \ldots, \vec{v}_4$ is a basis of $V$. Prove that 
    \[ v_1 + v_2, v_2+v_3, v_3+v_4, v_4 \]
    is also a basis of $V$.
    \item Prove or give a counterexample: If $\vec{v}_1, \ldots, \vec{v}_4$ is a basis of $V$ and $U$ is a
    subspace of $V$ such that $\vec{v}_1, \vec{v}_2\in U$ and $\vec{v}_3\not\in U$ and $\vec{v}_4\not\in U$, then $\vec{v}_1, \vec{v}_2$ is a basis of $U$.
    \item Recall $\mathcal{P}_m(\mathbb{R})$ is the space of polynomials with real coefficients and degree at most m. Show that if $p_0, \ldots, p_m\in \mathcal{P}_m(\mathbb{R})$ are such that $\deg(p_i) = i$, then $p_0, \ldots, p_m$ is a basis for $\mathcal{P}_m(\mathbb{R})$.
\end{enumerate}
\end{problem}

\begin{problem}[Dimesion]~
\begin{enumerate}
    \item Prove that if $X = Y_1 \oplus Y_2 \oplus \cdots \oplus Y_m$, then $\dim(X) = \dim(Y_1) + \dim(Y_2) + \cdots + \dim(Y_m)$.
    \item Recall $\mathcal{P}_4(\mathbb{R})$ is the space of polynomials with real coefficients and degree at most 4. 
    Let $U = \{ p \in \mathcal{P}_4(\mathbb{R}) : \int_{-1}^{1} p = 0\}$. 
    \begin{itemize}
        \item Find a basis of $U$
        \item Extend the previous basis to a basis for $\mathcal{P}_4(\mathbb{R})$.
        \item Find a subspace $W$ of $\mathcal{P}_4(\mathbb{R})$ such that $\mathcal{P}_4(\mathbb{R}) = U \oplus W$.
    \end{itemize}
    \item Suppose $U$ and $V$ are both 3 dimensional subspaces of $\mathbb{R}^5$. Prove that $U \cap V \neq \{ 0 \} $.
\end{enumerate}
\end{problem}


\end{document}

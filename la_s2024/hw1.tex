\documentclass[12pt]{article}

\usepackage[margin=1.25in]{geometry}

\usepackage{graphicx}
\usepackage{amsmath,amsthm,amssymb}

% you will probably have to remove the font-setup
\usepackage[no-math]{fontspec}
\setmainfont[
    BoldFont = Vollkorn Bold,
    ItalicFont = Vollkorn Italic,
    BoldItalicFont={Vollkorn Bold Italic},
    RawFeature=+lnum,
]{Vollkorn}

\setsansfont[
    BoldFont = Lato Bold,
    FontFace={l}{n}{*-Light},
    FontFace={l}{it}{*-Light Italic},
]{Lato}

\usepackage{unicode-math}
\mathitalicsmode=1

\setmathfont[
mathit = sym,
mathup = sym,
mathbf = sym,
math-style = TeX, 
bold-style = TeX
]{Wholegrain Math}
\everydisplay{\Umathoperatorsize\displaystyle=4ex}
\AtBeginDocument{\renewcommand\setminus{\smallsetminus}}
% font setup ends here


\theoremstyle{definition}
\newtheorem{problem}{Problem}
\newenvironment{solution}
  {\textbf{Solution.} }
  {}


\usepackage{enumitem}
\setlist[itemize]{itemsep=-.15em, topsep=-.5em}
\setlist[enumerate]{itemsep=-.15em, topsep=-.5em}
\setlist[enumerate,1]{label={(\alph*)}}

\usepackage{hyperref}

\setlength{\parskip}{1em}
\setlength{\parindent}{0em}

\renewcommand{\d}{\mathrm{d}}
\newcommand{\T}{\mathsf{T}}


\begin{document}

\textbf{\Large\sffamily{Homework 1\hfill Linear Algbera I}}
    
\vspace{-1.8em}
\hrulefill

\textbf{Instructions:}
    \begin{itemize}
        \item Due Feb 5 at 11:59pm on \href{https://www.gradescope.com/courses/709136}{Gradescope}.
        \item You must follow the submission policy in the \href{https://courses.chen.pw/la_s2024/syllabus.html}{syllabus} 
\end{itemize}
   
\vspace{.5em}

\begin{problem}[Complex Numbers]~
    \begin{enumerate}
        \item  Show that $\alpha +\beta = \beta + \alpha$ for all $\alpha,\beta \in \mathbb{C}$.
        \item Show that $\lambda (\alpha + \beta) = \lambda \alpha + \lambda \beta$ for all $\alpha, \beta,\lambda \in \mathbb{C}$.
    \end{enumerate}
\end{problem}

\begin{problem}[Vector Spaces]~
    \begin{enumerate}
        \item Let $V$ be a vector space. Prove that $-(-\vec{v}) = \vec{v}$ for all $\vec{v}\in V$.
        \item Prove $0\vec{v} = \vec{0}$ for all $\vec{v}\in V$.
        \item Let $V = \mathbb{R}^n$. Suppose $a\in \mathbb{R}$ and $v\in V$. Prove that if $a \vec{v} = \vec{0}$, then $a = 0$ or $\vec{v} = \vec{0}$.
        \item The empty set $\{\}$ is not a vector space. Which property of vector spaces is not satisfied by the empty set?
    \end{enumerate}
\end{problem}

\begin{problem}[Subspaces]~
\begin{enumerate}
    \item Prove that if $X$ and $Y$ are linear susbpaces of $V$, then so is $X+Y$.
\item Prove that the set $\{0\}$ consisting of the zero element of $V$ is a subspace of $V$.
\item Suppose $b\in\mathbb{R}$. 
Show that the set of continuous real-valued functions $f$ on the interval $[0,1]$ such that $\int_0^1 f(x) \d x = b$ is a subspace of the vector space of all continuous real-valued functions on $[0,1]$ if and only if $b=0$.
\end{enumerate}
\end{problem}

\begin{problem}
    Let $(V,+,\cdot,\mathbb{F})$ be a vector space. Suppose $U$ is a non-empty subset of $V$ closed under vector addition and scalar multiplication; i.e. for all $a,b\in\mathbb{F}$ and $\vec{u},\vec{v}\in V$, $a\vec{u} + b \vec{v} \in U$. 
    Prove that $(U,+,\cdot,\mathbb{F})$ is a vector space.
\end{problem}

% \begin{problem}[Span/Independence]~
% \begin{enumerate}
% \item Let $\vec{v}_1, \ldots, \vec{v}_k \in V$. Prove $\operatorname{span}(\vec{v}_1, \ldots, \vec{v}_k)$ is a subspace of $V$.
% \item Show that if the set of vectors $\vec{v}_1, \ldots, \vec{v}_k$ are linearly independent, then none of the $\vec{v}_i$ are the zero vector.
% \end{enumerate}
% \end{problem}

\end{document}

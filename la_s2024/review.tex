\documentclass[12pt]{article}

\usepackage[margin=.5in]{geometry}

% \usepackage[no-math]{fontspec}
% \setmainfont[
%     BoldFont = Vollkorn Bold,
%     ItalicFont = Vollkorn Italic,
%     BoldItalicFont={Vollkorn Bold Italic},
%     RawFeature=+lnum,
% ]{Vollkorn}

% \setsansfont[
%     BoldFont = Lato Bold,
%     FontFace={l}{n}{*-Light},
%     FontFace={l}{it}{*-Light Italic},
% ]{Lato}

\usepackage{graphicx}
\usepackage{amsmath,amsthm,amssymb}
% \usepackage{unicode-math}
% \mathitalicsmode=1

% \setmathfont[
% mathit = sym,
% mathup = sym,
% mathbf = sym,
% math-style = TeX, 
% bold-style = TeX
% ]{Wholegrain Math}
% \everydisplay{\Umathoperatorsize\displaystyle=4ex}
% \AtBeginDocument{\renewcommand\setminus{\smallsetminus}}


\newcommand\extrafootertext[1]{%
    \bgroup
    \renewcommand\thefootnote{\fnsymbol{footnote}}%
    \renewcommand\thempfootnote{\fnsymbol{mpfootnote}}%
    \footnotetext[0]{#1}%
    \egroup
}

\theoremstyle{definition}
\newtheorem{problem}{Problem}
\newenvironment{solution}
  {\textbf{Solution.} }
  {}

\usepackage{listings}
\lstset{basicstyle=\ttfamily,breaklines=true}

\usepackage{enumitem}
\setlist[itemize]{itemsep=-.15em, topsep=-.5em}
\setlist[enumerate]{itemsep=-.15em, topsep=-.5em}

\PassOptionsToPackage{hyphens}{url}
\usepackage{hyperref}

\setlength{\parskip}{1.5em}
\setlength{\parindent}{0em}

\newcommand{\range}{\operatorname{range}}
\renewcommand{\null}{\operatorname{null}}
\usepackage{arydshln}

\begin{document}

    \textbf{\Large\sffamily{Final Review Problems\hfill Linear Algebra I}}
    
    \vspace{-1.8em}
    \hrulefill
\vspace{.25em} 

The final will be in class April 30. 
The format and style will be similar to the quizzes, and you can use two pages of 8.5x11 notes.
All content covered through April 16 may appear on the final.

The following are some selected practice problems which are broadly representative of the type of problems that could appear (but not exhaustive).
You should also review the past quizzes, homeworks, and textbook problems.

\begin{problem}
    Recall that a non-empty subset $U$ of a vector space $V$ is a \emph{subspace} if it is closed under addition and scalar multiplication. 
    
    Let $V$ be the vector space of continuous real-valued functions on $[-5,5]$. 
    For $b\in \mathbb{R}$, define a subset $U$ of $V$ by $U = \{ f \in V : f'(0) + f''(3) = b \}$.
    Show that $U$ is a subspace of $V$ \emph{if and only if} $b=0$.
\end{problem}

\begin{problem}
    Suppose $\vec{v}_1, \ldots, \vec{v}_n$ is a linearly independent.
    Prove $\vec{v}_1, \ldots, \vec{v}_k$ is linearly independent if $k\leq n$.
\end{problem}


\begin{problem}
    Suppose $\vec{v}_1, \ldots, \vec{v}_n$ is a basis for a vector space $X$. 
    Let $Y = \text{span}(\vec{v}_1, \ldots, \vec{v}_k)$ for some $k \leq n$.
    Show that $\vec{v}_{1}, \ldots, \vec{v}_k$ is a basis for $Y$.
\end{problem}

\begin{problem}
    Let $v_1, \ldots, v_n$ be a linearly independent set of vectors. Suppose $v = c_1 v_1 + \cdots + c_n v_n$ and $v' = c_1' v_1 + \cdots + c_n' v_n$.
    Show that $v = v'$ if and only if $c_i = c_i'$ for all $i=1,\ldots,n$.
\end{problem}

\begin{problem}
    Suppose $U$ and $W$ are subspaces of $V$ such that $V=U \oplus W$. Suppose also that $u_1, \ldots, u_m$ is a basis of $U$ and $w_1, \ldots, w_n$ is a basis of $W$. Prove that
    \[
    u_1, \ldots, u_m, w_1, \ldots, w_n
    \]
    is a basis of $V$.
\end{problem}

\begin{problem}
    Suppose that $V$ is finite-dimensional and that $T \in \mathcal{L}(V, W)$. Prove that there exists a subspace $U$ of $V$ such that
    \[
    U \cap \operatorname{null} T=\{0\} \quad \text { and } \quad \operatorname{range} T=\{T u: u \in U\} .
    \]
\end{problem}

\begin{problem}
    Suppose $U$ and $V$ are finite-dimensional vector spaces and $S \in \mathcal{L}(V, W)$ and $T \in \mathcal{L}(U, V)$. Prove that
\[
\operatorname{dim} \text { null } S T \leq \operatorname{dim} \text { null } S+\operatorname{dim} \operatorname{null} T .
\]
\end{problem}

\begin{problem}
    Suppose $T \in \mathcal{L}(V, W)$ and $v_1, \ldots, v_m$ is a list of vectors in $V$ such that $T v_1, \ldots, T v_m$ is a linearly independent list in $W$. Prove that $v_1, \ldots, v_m$ is linearly independent.
\end{problem}

\begin{problem}
    Show that if $S, T \in \mathcal{L}(V, W)$, then $\mathcal{M}(S+T)=\mathcal{M}(S)+\mathcal{M}(T)$.
\end{problem}

\begin{problem}

\textbf{Fact 1.} If $q_0, \ldots, q_k\in\mathcal{P}_k$ are such that $\deg(q_i) = i$, then they are a basis for $\mathcal{P}_k$.

\textbf{Fact 2.} Let $T:V\to W$ be a linear map. The fundamental theorem of linear maps says if $\dim(V)<\infty$, then
\[
    \dim(V) = \dim(\null(T)) + \dim(\range(T)).
\]

\textbf{Fact 3.} Let $V$ be a finite dimensional vector space and $U$ a subspace of $V$. If $\dim(U) = \dim(v)$, then $U = V$.

    Suppose $T$ is a linear map from $\mathcal{P}_2$ to $\mathbb{R}^3$ with $\range(T) = \left\{ (a,b,0) : a,b\in \mathbb{R} \right\}$.
    Let $q\in\mathcal{P}_2$ be defined by $q(x) = x+1$ and suppose 
    \[
        T q = (0,0,0)
        .
    \]
    Find vectors $p_1,p_2\in\mathcal{P}_2$ so that $v_1 = Tp_1$ and $v_2 = Tp_2$ form a basis for $\range(T)$ (and prove it)

    \textbf{Hint}: Start by extending $q$ to a basis $q,p_1,p_2$ for $\mathcal{P}_2$. Then show $v_1$ and $v_2$ are linearly independent by considering $\alpha_1 v_1 + \alpha_2 v_2 = 0$ and using the definition of $v_1$ and $v_2$ and what you know about the null space of $T$. Finally, argue $v_1$ and $v_2$ are spanning.

\end{problem}

\begin{problem}
    Suppose $T$ is invertible and $S_1$ and $S_2$ are inverses of $T$. Prove $S_1 = S_2$.
\end{problem}

\begin{problem}
Suppose $T\in\mathcal{L}(U,V)$ and $S\in \mathcal{L}(V,W)$.
Prove that $\mathcal{M}(ST) = \mathcal{M}(S)\mathcal{M}(T)$.
\end{problem}

\begin{problem}
    We are used to working with the monomial basis:
    \[
    m_0(x) = 1, \quad
    m_1(x) = x, \quad
    m_2(x) = x^2, \quad
    m_3(x) = x^3, \quad
    m_4(x) = x^4
    \]
    Physicists like the Legendre polynomials:
    \[
    L_0(x) = 1 ,\quad
    L_1(x) = x, \quad
    L_2(x) = \frac{1}{2}(3x^2 - 1), \quad
    L_3(x) = \frac{1}{2}(5x^3 - 3x).
    \]
    (a) Let $D:\mathcal{P}_4\to\mathcal{P}_4$ be the differentiation operator: $Dp = p'$.
    What is $C = \mathcal{M}(D,\{m_i\},\{m_i\})$?
    
    \vspace{-.5em}
    (b) Let $p = 2L_3$. 
    What is $v = \mathcal{M}(p,\{L_i\})$?


    (c)     
    The change of basis matrices for monomial to Legendre and Legendre to monomial are given by:
    \[
    A = \mathcal{M}(I,\{L_i\},\{m_i\}) = 
    \begin{bmatrix}
        1 & 0 & -\frac{1}{2} & 0 \\
        0 & 1 & 0 & -\frac{3}{2} \\
        0 & 0 & \frac{3}{2} & 0 \\
        0 & 0 & 0 & \frac{5}{2} \\
    \end{bmatrix}
    ,\qquad
    B = \mathcal{M}(I,\{m_i\},\{L_i\}) = 
    \begin{bmatrix}
        1 & 0 & \frac{1}{3} & 0 \\
        0 & 1 & 0 & \frac{3}{5} \\
        0 & 0 & \frac{2}{3} & 0 \\
        0 & 0 & 0 & \frac{2}{5} \\
    \end{bmatrix}.
    \]
    What is the formula for $\mathcal{M}(Dp,\{L_i\})$ in terms of $A$,$B$,$C$ and $v$? Also compute the numerical entries. You can do it by matrix products or other means.

\end{problem}

\begin{problem}
    Suppose $u_1, \ldots, u_n$ is a basis for $U$ and $v_1, \ldots, v_m$ is a basis for $V$ (both over the same field).
    Write a basis for $U\times V$.
\end{problem}

\begin{problem}
    Let $V$ be a vector space and $U\subset V$. 
    Recall $v+U = \{v+u : u\in U\}$.

    Prove that $v+U = w+U$ if and only if $v-w\in U$.
\end{problem}



\begin{problem}
    
    \textbf{Fact.}
    Suppose $R$ is upper triangular ($r_{i,j} = 0$ if $i>j$).
    If $R$ is invertible, then the inverse $R^{-1}$ is also upper triangular.


    Suppose $v_1, \ldots, v_n$ and $u_1, \ldots, u_n$ are such that 
        \[
        \begin{bmatrix}
            | & | & & | \\
            v_1 & v_2 & \cdots & v_n \\
            | & | & & | \\
        \end{bmatrix}
        =
        \begin{bmatrix}
            | & | & & | \\
            u_1 & u_2 & \cdots & u_n \\
            | & | & & | \\
        \end{bmatrix}
        R
        ,\quad\text{where}\quad
        R = 
        \begin{bmatrix}
            r_{1,1} & r_{1,2} & \cdots & r_{1,n} \\
            0 & r_{2,2} & \cdots & r_{2,n} \\
            \vdots & \vdots & \ddots & \vdots \\
            0 & 0 & \cdots & r_{n,n} \\
        \end{bmatrix}.
        \]
        
        \begin{enumerate}
            \item[(a)] Show that
            $\operatorname{span}\{v_1, \ldots, v_k\}$ is a subset of $\operatorname{span}\{u_1, \ldots, u_k\}$ for each $k=1,2, \ldots, n$.
            \item[(b)] 
            Prove that if $R$ is invertible,  $\operatorname{span}\{v_1, \ldots, v_k\} = \operatorname{span}\{u_1, \ldots, u_k\}$ for each $k=1,2, \ldots, n$.
            \\This can be viewed as a converse to the homework problem 3(a).
        \end{enumerate}
        
   
\end{problem}

\begin{problem}
    \begin{enumerate}[label=(\alph*)]
        \item 
    Find an orthonormal basis for 
    \[
    \operatorname{span}\left( 
    \begin{bmatrix}
        3 \\ 0 \\ 0 \\0
    \end{bmatrix},
    \begin{bmatrix}
        1 \\ 0 \\ 2 \\0
    \end{bmatrix},
    \begin{bmatrix}
        0 \\ 0 \\ 0 \\-2
    \end{bmatrix},
     \right)
    \]
    \item Solve the least squares problem
    \[
        \min_{x,y,z} \left\| 
        \begin{bmatrix}
            3 &1&0 \\
            0 &0&0 \\
            0 &2&0 \\
            0 &0&-2
        \end{bmatrix}
        \begin{bmatrix}
            x\\y\\z
        \end{bmatrix}
        - \begin{bmatrix}
            1\\2\\3\\4
        \end{bmatrix}
        \right\|_2
    \]
    \end{enumerate}
\end{problem}

\begin{problem}
    Recall the Chebyshev polynomials
    \begin{align*}
        T_0(x) = 1
        ,\qquad T_1(x) = x
        ,\qquad T_2(x) = 2x^2 - 1
        ,\qquad T_3(x) = 4x^3 - 3x 
    \end{align*}
    satisfy the orthogonality condition:
    \begin{equation*}
        \int_{-1}^1 T_n(x)\,T_m(x)\,\frac{\mathrm{d}x}{\sqrt{1-x^2}} = 
        \begin{cases}
        0             & ~\text{ if }~ n \ne m, \\
        \pi           & ~\text{ if }~ n=m=0, \\
        \frac{\pi}{2} & ~\text{ if }~ n=m \ne 0.
        \end{cases}
    \end{equation*}

    Let $f(x) = \sin(x)$. Find a degree $3$ polynomial $p(x)$ and a function $g(x)$ so that $f(x) = p(x) + g(x)$ and 
    \[
        \int_{-1}^1 p(x) g(x) \frac{1}{\sqrt{1-x^2}}\mathrm{d}{x} = 0.
    \]
    You can express your functions in terms of the Chebyshev polynomials and leave the coefficients as integrals.
\end{problem}

\begin{problem}
    Suppose $T \in \mathcal{L}(V)$ and $U$ is a subspace of $V$.
        (i) Prove that if $U \subseteq$ null $T$, then $U$ is invariant under $T$.
        (ii) Prove that if range $T \subseteq U$, then $U$ is invariant under $T$.
\end{problem}

\begin{problem}
    Suppose $T\in\mathcal{L}(V)$. Then $\operatorname{null}(T)$ and $\operatorname{range}(T)$ are invariant under $T$.
\end{problem}

\begin{problem}
    Let $e_1, e_2, e_3, e_4$ be a basis for $V$.
    Suppose $T\in\mathcal{L}(V)$ is defined by
    \[
        Te_1 = e_1,\quad
        Te_2 = e_2,\quad
        Te_3= -2e_3,\quad
        Te_4 = 3e_4.
        \]

    \begin{enumerate}[label=(\alph*)]
        \item Write the matrix of $T$ with respect to $e_1, \ldots, e_4$.
        \item Find the minimial polynomial of $T$; i.e. the monic polynomial of minimal degree for which $p(T) = 0$. 
        Prove the polynomial you found is of minimal degree and unique.
    \end{enumerate}
\end{problem}

\begin{problem}
    Suppose $T$ is invertible. Prove that $E(\lambda,T) = E(1/\lambda,T)$ for all $\lambda\neq 0$.
\end{problem}

\begin{problem}
    Let $V$ be a vector space of $\mathbb{C}$ with an inner product $\langle \cdot, \cdot\rangle$.
    Recall an operator $T$ is called self-adjoint if $\langle Tu,v\rangle = {\langle u,Tv\rangle}$ for all $u,v\in V$.

    Prove that $T$ is self-adjoint if and only if $\langle Tv,v\rangle$ is real for all $v\in V$.
\end{problem}

\end{document}

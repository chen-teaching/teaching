\documentclass[12pt]{article}

\usepackage[margin=1.25in]{geometry}

\usepackage{graphicx}
\usepackage{amsmath,amsthm,amssymb}



\theoremstyle{definition}
\newtheorem{problem}{Problem}
\newenvironment{solution}
  {\textbf{Solution.} }
  {}


\usepackage{enumitem}
\setlist[itemize]{itemsep=-.15em, topsep=-.5em}
\setlist[enumerate]{itemsep=-.15em, topsep=-.5em}
\setlist[enumerate,1]{label={(\alph*)}}

\usepackage{hyperref}

\setlength{\parskip}{1em}
\setlength{\parindent}{0em}

\renewcommand{\d}{\mathrm{d}}
\newcommand{\T}{\mathsf{T}}
\newcommand{\range}{\operatorname{range}}
\renewcommand{\null}{\operatorname{null}}


\usepackage[]{xcolor}

\begin{document}

\textbf{\Large\sffamily{Homework 5\hfill Linear Algebra I}}
    
\vspace{-1.8em}
\hrulefill

\textbf{Instructions:}
    \begin{itemize}
        \item Due April 9 at 11:59pm on \href{https://www.gradescope.com/courses/709136}{Gradescope}.
        \item You must follow the submission policy in the \href{https://courses.chen.pw/la_s2024/syllabus.html}{syllabus} 
\end{itemize}
   
\vspace{.5em}

\begin{problem}[Product and quotient space]~
    \begin{enumerate}
        \item For a positive integer $m$, show that $V^m = \underbrace{V\times V \times \cdots \times V}_{m\text{times}}$ is isomorphic to $\mathcal{L}(\mathbb{F}^m,V)$. Do not assume $V$ is finite-dimensional.
        \item Suppose $A_1=v+U_1$ and $A_2=w+U_2$ for some $v, w \in V$ and some subspaces $U_1, U_2$ of $V$. Prove that the intersection $A_1 \cap A_2$ is either a translate of some subspace of $V$ or is the empty set.
        \item An \href{https://en.wikipedia.org/wiki/Equivalence_relation}{equivalence relation} is a binary relation that is reflexive, symmetric and transitive. Fix a subspace $U$ of $V$. Show that $v\sim w$ if and only if $v-w\in U$ is an equivalence relation on $V$.
        \item Briefly explain how the previous problem relates to translates.
        \item Suppose $U$ is a subspace of $V$ such that $V / U$ is finite-dimensional. Prove that $V$ is isomorphic to $U \times(V / U)$.
    \end{enumerate}
    \end{problem}

\begin{problem}[Duality]~
\begin{enumerate}
    \item Explain why each linear functional is surjective or is the zero map.
    \item Show that the dual map of the identity operator on $V$ is the identity operator
    on $V'$.
    \item Suppose $m\geq 0$. What is the dual basis of $\{1,x-5,(x-5)^2, \ldots, (x-5)^m\}$ in $\mathcal{P}_m$?
    \item Suppose $T \in \mathcal{L}(V, W)$ and $w_1, \ldots, w_m$ is a basis of range $T$. Hence for each $v \in V$, there exist unique numbers $\varphi_1(v), \ldots, \varphi_m(v)$ such that
    $$
    T v=\varphi_1(v) w_1+\cdots+\varphi_m(v) w_m,
    $$
    thus defining functions $\varphi_1, \ldots, \varphi_m$ from $V$ to $\mathbf{F}$. Show that each of the functions $\varphi_1, \ldots, \varphi_m$ is a linear functional on $V$.
\end{enumerate}
\end{problem}

\clearpage
\begin{problem}~

    \begin{enumerate}
        \item  Suppose $v_1, \ldots, v_n$ and $v_1, \ldots, u_n$ are such that $\operatorname{span}\{v_1, \ldots, v_k\} = \operatorname{span}\{u_1, \ldots, u_k\}$ for each $k$.
        problems
        \item 
    Suppose we have a independent set of vectors $v_1, \ldots, v_n$ and apply Gram-Schmidt to obtain an orthonormal set $u_1, \ldots, u_n$ such that 

        \begin{center}
        \fbox{
        \begin{minipage}{.8\textwidth}
        Set $u_1 = v_1 / \|v_1\|$.

        For $k = 2, \ldots, n$, set:
        $$
        \hat{u}_k = v_k - \langle v_k, u_1\rangle u_1 - \cdots - \langle v_k, u_{k-1}\rangle u_{k-1}.
        $$
        and 
        $$
        u_k = \hat{u}_k / \|\hat{u}_k\|.
        $$
        \end{minipage}
        }
        \end{center}

        \item Show that the upper triangular matrix $R$ you described in part (a) can be obtained from the coefficients computed by the Gram--Schmidt algorithm. 
        That is, that you get the matrix $R$ ``for free'' from the Gram--Schmidt algorithm.

    \end{enumerate}
\end{problem}

\begin{problem}
    Consider the vector space $\mathcal{P}_4$ of polynomials of degree at most $4$. 
    Define an inner product on $\mathcal{P}_4$ by
    $$ 
    \langle p,q\rangle = \int_{-1}^{1} p(x) q(x) \frac{1}{\sqrt{1-x^2}}\d{x},
    \qquad \forall p,q\in\mathcal{P}_4.
    $$

    \begin{enumerate}
        \item Verify this is an inner product.
        \item Apply the Gram-Schmidt process to the basis $\{1,x,x^2,x^3,x^4\}$ to obtain an orthonormal basis.
        You can use Wolfram alpha or similar to compute integrals, but should write down the integrals you are computing.
        \item Make a plot of the polynomials you computed and a different plot of the Chebyshev polynomials (up to degree 4). How do they compare?
    \end{enumerate}
\end{problem}

\clearpage
\begin{problem}
    \begin{enumerate}
        \item Suppose $V$ is a real inner product space and $v_1, \ldots, v_m$ is a linearly independent list of vectors in $V$. Prove that there exist exactly $2^m$ orthonormal lists $e_1, \ldots, e_m$ of vectors in $V$ such that
        $$
        \operatorname{span}\left(v_1, \ldots, v_k\right)=\operatorname{span}\left(e_1, \ldots, e_k\right)
        $$
        for all $k \in\{1, \ldots, m\}$.
        \item Suppose $C[-1,1]$ is the vector space of continuous real-valued functions on the interval $[-1,1]$ with inner product given by
        $$
        \langle f, g\rangle=\int_{-1}^1 f g
        $$
        for all $f, g \in C[-1,1]$. Let $\varphi$ be the linear functional on $C[-1,1]$ defined by $\varphi(f)=f(0)$. Show that there does not exist $g \in C[-1,1]$ such that
        $$
        \varphi(f)=\langle f, g\rangle
        $$
        for every $f \in C[-1,1]$.
        \item Suppose $V$ is finite-dimensional. Suppose $\langle\cdot, \cdot\rangle_1,\langle\cdot, \cdot\rangle_2$ are inner products on $V$ with corresponding norms $\|\cdot\|_1$ and $\|\cdot\|_2$. Prove that there exists a positive number $c$ such that $\|v\|_1 \leq c\|v\|_2$ for every $v \in V$.
    \end{enumerate}
\end{problem}

\end{document}

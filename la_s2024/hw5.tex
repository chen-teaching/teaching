\documentclass[12pt]{article}

\usepackage[margin=1.25in]{geometry}

\usepackage{graphicx}
\usepackage{amsmath,amsthm,amssymb}



\theoremstyle{definition}
\newtheorem{problem}{Problem}
\newenvironment{solution}
  {\textbf{Solution.} }
  {}


\usepackage{enumitem}
\setlist[itemize]{itemsep=-.15em, topsep=-.5em}
\setlist[enumerate]{itemsep=-.15em, topsep=-.5em}
\setlist[enumerate,1]{label={(\alph*)}}

\usepackage{hyperref}

\setlength{\parskip}{1em}
\setlength{\parindent}{0em}

\renewcommand{\d}{\mathrm{d}}
\newcommand{\T}{\mathsf{T}}
\newcommand{\range}{\operatorname{range}}
\renewcommand{\null}{\operatorname{null}}


\usepackage[]{xcolor}

\begin{document}

\textbf{\Large\sffamily{Homework 5\hfill Linear Algebra I}}
    
\vspace{-1.8em}
\hrulefill

{\color{red}More problems will be added}

\textbf{Instructions:}
    \begin{itemize}
        \item Due April 9 at 11:59pm on \href{https://www.gradescope.com/courses/709136}{Gradescope}.
        \item You must follow the submission policy in the \href{https://courses.chen.pw/la_s2024/syllabus.html}{syllabus} 
\end{itemize}
   
\vspace{.5em}

\begin{problem}[Product and quotient space]~
    \begin{enumerate}
        \item For a positive integer $m$, show that $V^m = \underbrace{V\times V \times \cdots \times V}_{m\text{times}}$ is isomorphic to $\mathcal{L}(\mathbb{F}^m,V)$. Do not assume $V$ is finite-dimensional.
        \item Suppose $A_1=v+U_1$ and $A_2=w+U_2$ for some $v, w \in V$ and some subspaces $U_1, U_2$ of $V$. Prove that the intersection $A_1 \cap A_2$ is either a translate of some subspace of $V$ or is the empty set.
        \item An \href{https://en.wikipedia.org/wiki/Equivalence_relation}{equivalence relation} is a binary relation that is reflexive, symmetric and transitive. Fix a subspace $U$ of $V$. Show that $v\sim w$ if and only if $v-w\in U$ is an equivalence relation on $V$.
        \item Briefly explain how the previous problem relates to translates.
        \item Suppose $U$ is a subspace of $V$ such that $V / U$ is finite-dimensional. Prove that $V$ is isomorphic to $U \times(V / U)$.
    \end{enumerate}
    \end{problem}

\begin{problem}[Duality]~
\begin{enumerate}
    \item Explain why each linear functional is surjective or is the zero map.
    \item Show that the dual map of the identity operator on $V$ is the identity operator
    on $V'$.
    \item Suppose $m\geq 0$. What is the dual basis of $\{1,x-5,(x-5)^2, \ldots, (x-5)^m\}$ in $\mathcal{P}_m$?
    \item Suppose $T \in \mathcal{L}(V, W)$ and $w_1, \ldots, w_m$ is a basis of range $T$. Hence for each $v \in V$, there exist unique numbers $\varphi_1(v), \ldots, \varphi_m(v)$ such that
    $$
    T v=\varphi_1(v) w_1+\cdots+\varphi_m(v) w_m,
    $$
    thus defining functions $\varphi_1, \ldots, \varphi_m$ from $V$ to $\mathbf{F}$. Show that each of the functions $\varphi_1, \ldots, \varphi_m$ is a linear functional on $V$.
\end{enumerate}
\end{problem}

\end{document}

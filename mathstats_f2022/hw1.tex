\documentclass{article}

\usepackage[margin=1.25in]{geometry}
\usepackage{amsmath,amsthm,amssymb}

\newcommand\extrafootertext[1]{%
    \bgroup
    \renewcommand\thefootnote{\fnsymbol{footnote}}%
    \renewcommand\thempfootnote{\fnsymbol{mpfootnote}}%
    \footnotetext[0]{#1}%
    \egroup
}

\newtheorem{problem}{Problem}
\newenvironment{solution}
  {\textbf{Solution.} }
  {}

\usepackage{enumitem}
\usepackage{hyperref}

\setlength{\parskip}{1em}
\setlength{\parindent}{0em}


\newcommand{\PP}{\mathbb{P}}
\newcommand{\EE}{\mathbb{E}}
\newcommand{\VV}{\mathbb{V}}


\begin{document}

\begin{center}
    \textbf{\Large{Mathematical Statistics (MATH-UA 234)}}
 
    \textbf{\Large{Homework 1}}

    Due 09/08 at the beginning of class on \href{https://www.gradescope.com/courses/414277/}{Gradescope}
\end{center}


\extrafootertext{problems with a textbook reference are based on, but not identical to, the given reference}

\vspace{2em}


\begin{problem}[Wasserman 1.19 (Bayes theorem)]
Suppose that 30 percent of computer owners use a Macintosh, 50 percent
use Windows, and 20 percent use Linux. Suppose that 65 percent of
the Mac users have succumbed to a computer virus, 82 percent of the
Windows users get the virus, and 50 percent of the Linux users get
the virus. We select a person at random and learn that her system was
infected with the virus. What is the probability that they are a Windows
user?
\end{problem}

\begin{problem}
    Suppose $\PP[A] = 2/3$ and $\PP[B^c] = 1/4$. Can $A$ and $B$ be disjoint (only one can occur)? Why or why not?
\end{problem}


\begin{problem}
    Use the axioms of a probability distribution to show or answer the following:
    \begin{enumerate}[label=(\alph*),topsep=0pt]
        \item Show $\PP[\emptyset] = 0$.
        \item Show $A\subseteq B \Longrightarrow \PP[A] \leq \PP[B]$.
        \item Does $A\subsetneq B$ imply $\PP[A] < \PP[B]$?
        \item Show $0\leq \PP[A] \leq 1$ for all $A$.
    \end{enumerate}

\end{problem}

\begin{problem}
    Use the axioms of a probability distribution to answer the following:
    \begin{enumerate}[label=(\alph*),topsep=0pt]
        \item Describe a situation where we might have a sample space $\Omega$ such that $\PP[\{\omega\}] = 0$ for all $\omega\in\Omega$.
        \item 
            Explain what is wrong with the following proof that $1=0$.
            
            \textbf{Proof.} By definition $\cup_{\omega \in \Omega} \{\omega\} = \{ \omega : \exists \omega' \in \Omega \text{ with } \omega \in \{\omega' \} \} = \Omega$.
            Thus,
            \begin{align*}
                1 &= \PP[\Omega] \tag{Axiom 2}
                \\&= \PP\bigg[\bigcup_{\omega \in \Omega} \{ \omega\}\bigg]  \tag{definition of union}
                \\&= \sum_{\omega \in \Omega} \PP[\{\omega\} ] \tag{Axiom 3} 
                \\&= \sum_{\omega \in \Omega} 0  \tag{assumption} 
                \\&= 0 \tag*{\qed}
            \end{align*}
    \end{enumerate}
\end{problem}

\begin{problem}[Wasserman 1.22 (simulate coin fipping)]
Suppose we flip a coin $n$ times and let $p$ denote
the probability of heads. Let $X$ be the number of heads. We call $X$
a binomial random variable, which is discussed in the next chapter.
Intuition suggests that $X$ will be close to $np$. To see if this is true, we
can repeat this experiment many times and average the $X$ values. Carry
out a simulation and compare the average of the $X$'s to $np$. 

Try this for
$p =.3$ and $n = 10$, $n = 100$, and $n = 1000$ repeating each experiment 100 and then 1000 times.
\end{problem}

\begin{problem}[Wasserman 2.2 (computing probabilities from distributions)]
    Let $X$ be such that $\PP[X=2] = \PP[X=3] = 1/10$ and $\PP[X=5] = 8/10$.

    Plot two different possible CDFs $F$ for $X$ and use these CDFs to find $\PP[2<X<4.8]$ and $\PP[2\leq X\leq 4.8]$.

\end{problem}

\begin{problem}[Wasserman 2.17 (conditional probabilities from distributions)]
Let
    \begin{equation*}
        f_{X,Y}(x,y) 
        = 
        \begin{cases}
            c(x+y^2) & 0\leq x\leq 1 \text{ and } 0\leq y \leq 1 \\
            0 & \text{otherwise.}
        \end{cases}
    \end{equation*}
    Find $\PP[X<1/2 | Y = 1/2]$. Here $c$ is a normalizing constant so that $f_{X,Y}$ is a probability density function.
\end{problem}

\begin{problem}[Wasserman 2.21 (important trick about iid maxes)]
    Let $X_1, \ldots, X_n\sim\operatorname{Exp}(\eta)$ be iid. Let $Y=\max\{X_1, \ldots, X_n\}$. Find the PDF of $Y$. Hint: $Y\leq y$ if and only if $X_i \leq y$ for $i=1,\ldots, n$.
\end{problem}

\begin{problem}
    Look for an instance of one of the probability concepts we've seen in the course which you noticed in a different part of your life (e.g. in other classes, on the subway, at the park, on TV, etc.).

    In a few sentences, explain this instance and how it could be described mathematically.
\end{problem}

\end{document}

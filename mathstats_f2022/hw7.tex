\documentclass{article}

\usepackage[margin=1.25in]{geometry}


\usepackage{graphicx}
\usepackage{amsmath,amsthm,amssymb}
\usepackage{unicode-math}
\mathitalicsmode=1

\setmathfont[
mathit = sym,
mathup = sym,
mathbf = sym,
math-style = TeX, 
bold-style = TeX
]{Wholegrain Math}
\everydisplay{\Umathoperatorsize\displaystyle=4ex}
\AtBeginDocument{\renewcommand\setminus{\smallsetminus}}


\newcommand\extrafootertext[1]{%
    \bgroup
    \renewcommand\thefootnote{\fnsymbol{footnote}}%
    \renewcommand\thempfootnote{\fnsymbol{mpfootnote}}%
    \footnotetext[0]{#1}%
    \egroup
}

\newtheorem{problem}{Problem}
\newenvironment{solution}
  {\textbf{Solution.} }
  {}

\usepackage{listings}
\lstset{basicstyle=\ttfamily,breaklines=true}

\usepackage{enumitem}
\PassOptionsToPackage{hyphens}{url}
\usepackage{hyperref}

\setlength{\parskip}{1em}
\setlength{\parindent}{0em}

\renewcommand{\d}{\mathrm{d}}

\newcommand{\PP}{\mathbb{P}}
\newcommand{\EE}{\mathbb{E}}
\newcommand{\VV}{\mathbb{V}}
\newcommand{\CoV}{\operatorname{Co\mathbb{V}}}
\usepackage{dsfont}
\newcommand{\bOne}{\mathds{1}}

\begin{document}

\begin{center}
    \textbf{\Large\sffamily{Homework 7: Mathematical Statistics (MATH-UA 234)}}
 

    Due 12/08 at the beginning of class on \href{https://www.gradescope.com/courses/414277/}{Gradescope}.
    The quiz will still be 12/06, and will cover content from problems 1-3 (i.e. Bayesian inference). No solutions will be posted prior to the quiz.
\end{center}


\extrafootertext{\hspace{-14.2pt}problems with a textbook reference are based on, but not identical to, the given reference}

\vspace{2em}


\textbf{Reminder.} 
Remember than the project presentations are on December 14th!

\begin{problem}
    Suppose $X_1, \ldots, X_n \sim \operatorname{Ber}(p)$ (with $1$ representing heads and zero representing tails) and that we use the prior distribution $p\sim \operatorname{Beta}(\alpha,\beta)$.
    \begin{enumerate}[label=(\alph*),topsep=0pt]
        \item Compute the posteriori distribution for $p|X_1=x_1, \ldots, X_n=x_n$.
        \item For each of the coins below, find values of $\alpha$ and $\beta$ so that your prior distribution represents your belief about the parameter $p$ of the coin.
            Plot and label these 6 prior distributions.
            Note that the head side is the side marked with the number.
        \item Suppose you flipped coin zero and got 53 heads and 47 tails. Make a plot showing the prior and posterior densities for $p$.
        \item Suppose you flipped coin 4 and got 39 heads and 61 tails. Make a plot showing the prior and the posterior densities for $p$.
        \item Suppose you flipped coin 6 and got 0 heads and 100 tails. Make a plot showing the prior and the posterior densities for $p$.
        \item For the coin 6 example, is the probability that $p=0$ under your posterior 100\%? 
        Does this make sense? Why or why not?
    \end{enumerate}
    \includegraphics[width=\textwidth]{coins-all.jpg}
    This image was taken from this site: \url{https://izbicki.me/blog/how-to-create-an-unfair-coin-and-prove-it-with-math.html}
\end{problem}

\begin{problem}[Wasserman 11.1]
    Suppose $X_1, \ldots, X_n \sim N(\theta,\sigma^2)$, and that we use the prior distribution $\theta\sim N(a,b^2)$.
    Show that $\theta|X_1=x_1, \ldots, X_n=x_n \sim N(\bar{\theta},\tau^2)$ where 
    \begin{equation*}
        \bar{\theta} = w \frac{x_1+ \cdots + x_n}{n} + (1-w)a
        ,\qquad w = \frac{1/\mathsf{se}^2}{1/\mathsf{se}^2 + 1/b^2}
        ,\qquad \tau = 1/\sqrt{1/\mathsf{se}^2 + 1/b^2}
        ,\qquad \mathsf{se} = \sigma/\sqrt{n}.
    \end{equation*}
\end{problem}

\clearpage
\begin{problem}[Wasserman 11.2]
    Let $X_1, \ldots, X_n\sim N(\mu,1)$.
    \begin{enumerate}[label=(\alph*),topsep=0pt]
        \item Simulate a dataset (using $\mu=5$) consisting of $n=100$ observations
        \item Take $f(\mu) = 1$ as the prior density, and find the posterior density given the observed data. Plot this density
    \end{enumerate}
\end{problem}


\begin{problem}
    Consider a model of the form $f(x) = \hat{\beta}_0 + \hat{\beta}_1 y$ and, given data 
    \begin{equation*}
        (X_1, Y_1), \ldots, (X_n, Y_n),
    \end{equation*}
    define the loss function
    \begin{equation*}
        L(\hat{\beta}_0, \hat{\beta}_1) = \sum_{i=1}^{n} (Y_i - f(X_i))^2. 
    \end{equation*}

    \begin{enumerate}[label=(\alph*),topsep=0pt]
        \item Compute the partial derivatives 
            $\partial L(\hat{\beta}_0,\hat{\beta}_1) / \partial \hat{\beta}_0$
            and $\partial L(\hat{\beta}_0,\hat{\beta}_1) / \partial \hat{\beta}_1$
        \item Find the minimizers $\hat{\beta}_0$ and $\hat{\beta}_1$ for $L(\hat{\beta}_0, \hat{\beta}_1)$.
        \item Show that you can write the loss function in the form $\|\vec{b} - \vec{A}\vec{x}\|_2^2$, where $\vec{b}$ is a particular vector of length $n$, $\vec{A}$ is a $n\times 2$ matrix, and $\vec{x}$ is a length $2$ vector.
    \end{enumerate}

\end{problem}

\begin{problem}
Consider the following four data sets:

    \begin{lstlisting}
x1 = [10, 8, 13, 9, 11, 14, 6, 4, 12, 7, 5]
y1 = [8.04, 6.95, 7.58, 8.81, 8.33, 9.96, 7.24, 4.26, 10.84, 4.82, 5.68]

x2 = [10, 8, 13, 9, 11, 14, 6, 4, 12, 7, 5]
y2 = [9.14, 8.14, 8.74, 8.77, 9.26, 8.10, 6.13, 3.10, 9.13, 7.26, 4.74]

x3 = [10, 8, 13, 9, 11, 14, 6, 4, 12, 7, 5]
y3 = [7.46, 6.77, 12.74, 7.11, 7.81, 8.84, 6.08, 5.39, 8.15, 6.42, 5.73]

x4 = [8, 8, 8, 8, 8, 8, 8, 19, 8, 8, 8]
y4 = [6.58, 5.76, 7.71, 8.84, 8.47, 7.04, 5.25, 12.50, 5.56, 7.91, 6.89]

    \end{lstlisting}

\begin{enumerate}[label=(\alph*),topsep=0pt]
    \item Find the sample mean and sample variance of each datasets' $X$ and $Y$ values. Compute the sample correlation between the datasets.
    \item Find the linear regression line and compute the $R^2$ value for each dataset.
    \item Now, plot the datasets and the linear regression lines. Explain what happened.
\end{enumerate}
\end{problem}

\begin{problem}[Wasserman 13.2]
    Suppose $Y_i = \beta_0 + \beta_1 X_i + \epsilon_i$, where $\EE[\epsilon_i | X_i] = 0$ and $\VV[\epsilon_i | X_i] = \sigma^2$.

    Let $\hat{\beta}_0$ and $\hat{\beta}_1$ be the least squares estimates given in Theorem 13.4.
%\begin{enumerate}[label=(\alph*),topsep=0pt]
    Show that $\EE[\hat{\beta}_0|X_1, \ldots, X_n] = \beta_0$ and $\EE[\hat{\beta}_1|X_1, \ldots, X_n] = \beta_1$.
    %\item Show that $\VV[\hat{\beta}_0 | X_1, \ldots, X_n] = \frac{\sigma^2}{n s_X} \frac{1}{n} \sum_{i=1}^{n} X_i^2$ and $\VV[\hat{\beta}_1 | X_1, \ldots, X_n] = \frac{\sigma^2}{n s_X}$, where $s_X = \frac{1}{n} \sum_{i=1}^{n}(X_i - \bar{X}_i)^2$.
%\end{enumerate}
    You should regard $X_1, \ldots, X_n$ as constant.
\end{problem}




\begin{problem}
    Pick at least one of the following articles to read. 
    Provide a one paragraph summary of what you think the most important points of the article were.
    Discuss how this is relevant to what we are learning in class.

    \begin{itemize}[nolistsep]
        \item \href{https://www.vox.com/recode/2020/2/18/21121286/algorithms-bias-discrimination-facial-recognition-transparency}{Why algorithms can be racist and sexist}
        \item \href{https://hbr.org/2019/05/all-the-ways-hiring-algorithms-can-introduce-bias}{All the Ways Hiring Algorithms Can Introduce Bias}
        \item \href{https://sitn.hms.harvard.edu/flash/2020/racial-discrimination-in-face-recognition-technology/}{Racial Discrimination in Face Recognition Technology}
    \end{itemize}
\end{problem}



\end{document}

\documentclass[12pt]{article}

\usepackage[margin=1.25in]{geometry}

\usepackage{graphicx}
\usepackage{amsmath,amsthm,amssymb}

% you will probably have to remove the font-setup
\usepackage[no-math]{fontspec}
\setmainfont[
    BoldFont = Vollkorn Bold,
    ItalicFont = Vollkorn Italic,
    BoldItalicFont={Vollkorn Bold Italic},
    RawFeature=+lnum,
]{Vollkorn}

\setsansfont[
    BoldFont = Lato Bold,
    FontFace={l}{n}{*-Light},
    FontFace={l}{it}{*-Light Italic},
]{Lato}

\usepackage{unicode-math}
\mathitalicsmode=1

\setmathfont[
mathit = sym,
mathup = sym,
mathbf = sym,
math-style = TeX, 
bold-style = TeX
]{Wholegrain Math}
\everydisplay{\Umathoperatorsize\displaystyle=4ex}
\AtBeginDocument{\renewcommand\setminus{\smallsetminus}}
% font setup ends here

\newcommand\extrafootertext[1]{%
    \bgroup
    \renewcommand\thefootnote{\fnsymbol{footnote}}%
    \renewcommand\thempfootnote{\fnsymbol{mpfootnote}}%
    \footnotetext[0]{#1}%
    \egroup
}

\theoremstyle{definition}
\newtheorem{problem}{Problem}
\newenvironment{solution}
  {\textbf{Solution.} }
  {}

\usepackage{listings}
\lstset{basicstyle=\ttfamily,breaklines=true}

\usepackage{enumitem}
\setlist[itemize]{itemsep=-.15em, topsep=-.5em}
\setlist[enumerate]{itemsep=-.15em, topsep=-.5em}
\setlist[enumerate,1]{label={(\alph*)}}

\PassOptionsToPackage{hyphens}{url}
\usepackage{hyperref}

\setlength{\parskip}{1em}
\setlength{\parindent}{0em}

\renewcommand{\d}{\mathrm{d}}
\renewcommand{\vec}{\mathbf}
\newcommand{\T}{\mathsf{T}}

\usepackage{dsfont}
\newcommand{\bOne}{\mathds{1}}


\begin{document}

    \textbf{\Large\sffamily{Homework 1\hfill Numerical Analysis Spring 2023}}
    
    \vspace{-1.8em}
    \hrulefill
 
\textbf{Instructions:}
    \begin{itemize}
        \item Due 02/02 at 11:59pm on \href{https://www.gradescope.com/courses/487363/}{Gradescope}.
        \item Write the names of anyone you work with on the top of your assignment. If you worked alone, write that you worked alone.
        \item Show your work.
        \item Include all code you use as copyable \verb|monospaced| text in the PDF (i.e. not as a screenshot).
        \item Do not put the solutions to multiple problems on the same page.
        \item Tag your responses on gradescope. Each page should have a \emph{single} problem tag. Improperly tagged responses will not receive credit.
\end{itemize}
    
\vspace{1em}

\begin{problem}~
    \begin{enumerate}
        \item The Summit supercomputer can do $10^{18}$ floating point operations per second.
            The supercomputer is physically large. 
            Suppose that the furthest nodes are separated by 100 meters.
            Roughly how many floating point operations could be done in the time it takes to send a signal from one of these node to the other? (You can assume the signal travels at the speed of light)
            \begin{center}
                \includegraphics[width=.7\textwidth]{summit.jpg}
                
                Image of Summit from \href{https://www.flickr.com/photos/olcf/49912253043/in/album-72157683655708262/}{ORNL}.
            \end{center}
        \item Suppose $\vec{A}$ is both an upper triangular matrix and a lower triangular matrix. Show that $\vec{A}$ is diagonal.
        \item Suppose $\vec{A}$ is a matrix. Show that $\vec{A}^\T \vec{A}$ is symmetric.
        \item Suppose $\vec{A}$ is a diagonal matrix and $\vec{B}$ is upper triangular. Show that $\vec{A}\vec{B}$ is upper triangular.
    \end{enumerate}
\end{problem}

\clearpage

\begin{problem}~
    Consider linear equations $f(x) = a x + b$ and $\tilde{f}(x) = a x + (b+\epsilon)$, where $a$ and $b$ are constants and $\epsilon = 10^{-5}$.
    \begin{enumerate}
        \item Find the solutions to $f(x) = 0$ and $\tilde{f}(x) = 0$ in terms of $a$ and $b$.
        \item Find values of $a$ and $b$ so that the solutions to these two equations are very different. Here ``very different'' means the solutions differ by much more than $\epsilon$.
        \item Using matplotlib, plot the functions $f(x)$ and $\tilde{f}(x)$ corresponding to your values of $a$ and $b$.
            Make sure the plots are labeled, and that the axes are chosen such that the two lines can be clearly differentiated from one another and the zeros are clearly visible.
    \end{enumerate}
\end{problem}


\begin{problem}
Define 
\begin{equation*}
    \vec{A} = \begin{bmatrix} 1 & 2 & 3 \\ 4 & 5 & 6 \\ 7 & 8 & 9 \\ 10 & 11 & 12\end{bmatrix}
    ,\quad
    \vec{B} = \begin{bmatrix} 0 & -1 & 4 \\ -7 & 0 & 1 \\ 3 & 4 & -3\end{bmatrix}
    ,\quad
    \vec{x} = \begin{bmatrix} 3 \\ 2 \\ -2\end{bmatrix}.
\end{equation*}

    \begin{enumerate}
        \item Compute $\vec{A} \vec{B}$.
        \item Compute $\vec{B} \vec{x}$.
        \item Compute $\vec{A} (\vec{B} \vec{x})$.
        \item Compute $(\vec{A}\vec{B}) \vec{x}$.
    \end{enumerate}
Show your work at each step.
\end{problem}

\begin{problem}~
\begin{enumerate}
    \item Make the matrices $\vec{A}$, $\vec{B}$, and vector $\vec{x}$ in numpy.
    \item Verify each of the answers you generated in Problem 1 using numpy. 
\end{enumerate}
\end{problem}

\end{document}

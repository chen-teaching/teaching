\documentclass[12pt]{article}

\usepackage[margin=1.25in]{geometry}

\usepackage{graphicx}
\usepackage{amsmath,amsthm,amssymb}

% you will probably have to remove the font-setup
\usepackage[no-math]{fontspec}
\setmainfont[
    BoldFont = Vollkorn Bold,
    ItalicFont = Vollkorn Italic,
    BoldItalicFont={Vollkorn Bold Italic},
    RawFeature=+lnum,
]{Vollkorn}

\setsansfont[
    BoldFont = Lato Bold,
    FontFace={l}{n}{*-Light},
    FontFace={l}{it}{*-Light Italic},
]{Lato}

\usepackage{unicode-math}
\mathitalicsmode=1

\setmathfont[
mathit = sym,
mathup = sym,
mathbf = sym,
math-style = TeX, 
bold-style = TeX
]{Wholegrain Math}
\everydisplay{\Umathoperatorsize\displaystyle=4ex}
\AtBeginDocument{\renewcommand\setminus{\smallsetminus}}
% font setup ends here

\newcommand\extrafootertext[1]{%
    \bgroup
    \renewcommand\thefootnote{\fnsymbol{footnote}}%
    \renewcommand\thempfootnote{\fnsymbol{mpfootnote}}%
    \footnotetext[0]{#1}%
    \egroup
}

\theoremstyle{definition}
\newtheorem{problem}{Problem}
\newenvironment{solution}
  {\textbf{Solution.} }
  {}

\usepackage{listings}
\lstset{basicstyle=\ttfamily,breaklines=true}

\usepackage{enumitem}
\setlist[itemize]{itemsep=-.15em, topsep=-.5em}
\setlist[enumerate]{itemsep=-.15em, topsep=-.5em}
\setlist[enumerate,1]{label={(\alph*)}}

\PassOptionsToPackage{hyphens}{url}
\usepackage{hyperref}

\setlength{\parskip}{1em}
\setlength{\parindent}{0em}

\renewcommand{\d}{\mathrm{d}}
\renewcommand{\vec}{\mathbf}
\newcommand{\T}{\mathsf{T}}

\usepackage{dsfont}
\newcommand{\bOne}{\mathds{1}}

\newcommand{\F}{\mathsf{F}}

\usepackage{array,booktabs}

\begin{document}

    \textbf{\Large\sffamily{Homework 6\hfill Numerical Analysis Spring 2023}}
    
    \vspace{-1.8em}
    \hrulefill
 
\textbf{Instructions:}
    \begin{itemize}
        \item Due 04/20 at 11:59pm on \href{https://www.gradescope.com/courses/487363/}{Gradescope}.
        \item Write the names of anyone you work with on the top of your assignment. If you worked alone, write that you worked alone.
        \item Show your work.
        \item Include all code you use as copyable \verb|monospaced| text in the PDF (i.e. not as a screenshot).
        \item Do not put the solutions to multiple problems on the same page.
        \item Tag your responses on gradescope. Each page should have a \emph{single} problem tag. Improperly tagged responses will not receive credit.
\end{itemize}
    
\vspace{2em}

\begin{problem}
    Spend at least two hours working on your project prior to the 20th. 
    Answer the following:
    \begin{enumerate}
        \item What is the current status of your project? 
        \item What are the big tasks you have left to do before your project is done?
        \item What is your plan for completing the project in a timely manner?
    \end{enumerate}
\end{problem}

\begin{problem}
    Suppose $\vec{A}$ has SVD $\vec{A} = \vec{U} \vec{\Sigma} \vec{V}^\T$ where 
    \[
        \vec{U} = 
        \begin{bmatrix}
            |&|&&|\\
            \vec{u}_1 & \vec{u}_2 & \cdots & \vec{u}_n\\
            |&|&&|\\
        \end{bmatrix},
        \qquad
        \vec{\Sigma} = \begin{bmatrix} \sigma_1 \\ &\sigma_2 \\ &&\ddots \\ &&& \sigma_n\end{bmatrix}
        ,\qquad
        \vec{V}
    =   \begin{bmatrix}
            |&|&&|\\
            \vec{v}_1 & \vec{v}_2 & \cdots & \vec{v}_n\\
            |&|&&|\\
        \end{bmatrix}. 
    \]
    \begin{enumerate}
        \item Show that $\vec{v}_i$ is an eigenvector of $\vec{A}^\T \vec{A}$. What is the corresponding eigenvalue?

        \item

    Define the block matrix:
    \[
        \vec{B} = 
        \begin{bmatrix} \vec{0}& \vec{A} \\ 
            \vec{A}^\T & \vec{0}
        \end{bmatrix}.
    \]
    Show that 
    \[
        \vec{x} = \begin{bmatrix} \vec{u}_i \\ \vec{v}_i \end{bmatrix}
    \]
    is an eigenvector of $\vec{B}$.
    What is the corresponding eigenvalue?
    \end{enumerate}
\end{problem}

\clearpage
\begin{problem}
Suppose $\vec{A}$ has eigenvalue decomposition:
\[
    \vec{A} 
    = 
    \vec{V}
    \begin{bmatrix}
        -4 \\ & -1 \\ && 2 \\ &&& 3
    \end{bmatrix}
    \vec{V}^{-1}
    ,\qquad
    \vec{V} = \begin{bmatrix} |&|&|&| \\ \vec{v}_1 & \vec{v}_2 & \vec{v}_3 & \vec{v}_4 \\ |&|&|&| \end{bmatrix}
    \]
where the $\vec{v}_i$ are all orthonormal.

    Suppose we run inverse power method with shift $c$ with $\vec{x} = \vec{v}_1 + \vec{v}_2 + \vec{v}_3 + \vec{v}_4$; that is, power method on $(\vec{A} - c\vec{I})^{-1}$.

    If $c\in(0.5,2.5)$, then we will converge to $\vec{v}_3$, the eigenvector corresponding to eigenvalue $2$.
    The rate of converge is
    \[
        \rho = \left| \frac{\lambda_2((\vec{A}-c\vec{I})^{-1})}{\lambda_1((\vec{A}-c\vec{I})^{-1})} \right|,
    \]
    where $\lambda_1((\vec{A}-c\vec{I})^{-1})$ and $\lambda_2((\vec{A}-c\vec{I})^{-1})$ are the largest and second largest eigenvalues of $(\vec{A} - c\vec{I})^{-1}$ in magnitude respectively.


    \begin{enumerate}
        \item Plot $\rho$ as a function of $c$ for $c$ in the range $(0.5, 2.5)$.

        \item Let $\vec{y}_k$ be the output of $k$-steps of the power method, and assume $\|\vec{v}_3 - \vec{y}_k\|_2 \leq \rho^k$.
            
            For $\epsilon = 10^{-1}$, make a plot showing how large $k$ has to be so that $\|\vec{v}_3 - \vec{y}_k\|_2 < \epsilon$ for the values of $c$ in the range $(0.5,2.5)$.
            Add more a new line to this plot for each $\epsilon = 10^{-2}, 10^{-5}, 10^{-10}$ plot. Label all the lines.

            

    \end{enumerate}




\end{problem}

\begin{problem}
For the same matrix as in Problem 3, suppose we run power method with a starting vector:
\[
\vec{x} = \vec{V}\begin{bmatrix}0\\1\\1\\1\end{bmatrix}.
\]

    \begin{enumerate}
        \item Find a vector $\vec{z}$ so that $\vec{A}^k \vec{x} = \vec{V} \vec{z}$.
        
        \item What vector does $\vec{z} / \|\vec{z}\|$ converge to as $k\to\infty$?

        \item What vector does $\vec{A}^k\vec{x} / \|\vec{A}^k\vec{x}\|$ converge to as $k\to\infty$?

        \item Why did we get something different than on worksheet 8, where power method converged to a multiple of $\vec{v}_1$?
    
    \end{enumerate}
\end{problem}


\end{document}

\documentclass[12pt]{article}

\usepackage[margin=1.25in]{geometry}

\usepackage{graphicx}
\usepackage{amsmath,amsthm,amssymb}

% you will probably have to remove the font-setup
\usepackage[no-math]{fontspec}
\setmainfont[
    BoldFont = Vollkorn Bold,
    ItalicFont = Vollkorn Italic,
    BoldItalicFont={Vollkorn Bold Italic},
    RawFeature=+lnum,
]{Vollkorn}

\setsansfont[
    BoldFont = Lato Bold,
    FontFace={l}{n}{*-Light},
    FontFace={l}{it}{*-Light Italic},
]{Lato}

\usepackage{unicode-math}
\mathitalicsmode=1

\setmathfont[
mathit = sym,
mathup = sym,
mathbf = sym,
math-style = TeX, 
bold-style = TeX
]{Wholegrain Math}
\everydisplay{\Umathoperatorsize\displaystyle=4ex}
\AtBeginDocument{\renewcommand\setminus{\smallsetminus}}
% font setup ends here

\newcommand\extrafootertext[1]{%
    \bgroup
    \renewcommand\thefootnote{\fnsymbol{footnote}}%
    \renewcommand\thempfootnote{\fnsymbol{mpfootnote}}%
    \footnotetext[0]{#1}%
    \egroup
}

\theoremstyle{definition}
\newtheorem{problem}{Problem}
\newenvironment{solution}
  {\textbf{Solution.} }
  {}

\usepackage{listings}
\lstset{basicstyle=\ttfamily,breaklines=true}

\usepackage{enumitem}
\setlist[itemize]{itemsep=-.15em, topsep=-.5em}
\setlist[enumerate]{itemsep=-.15em, topsep=-.5em}
\setlist[enumerate,1]{label={(\alph*)}}

\PassOptionsToPackage{hyphens}{url}
\usepackage{hyperref}

\setlength{\parskip}{1em}
\setlength{\parindent}{0em}

\renewcommand{\d}{\mathrm{d}}
\renewcommand{\vec}{\mathbf}
\newcommand{\T}{\mathsf{T}}

\usepackage{dsfont}
\newcommand{\bOne}{\mathds{1}}

\usepackage{array,booktabs}

\begin{document}

    \textbf{\Large\sffamily{Homework 2\hfill Numerical Analysis Spring 2023}}
    
    \vspace{-1.8em}
    \hrulefill
 
\textbf{Instructions:}
    \begin{itemize}
        \item Due 02/16 at 11:59pm on \href{https://www.gradescope.com/courses/487363/}{Gradescope}.
        \item Write the names of anyone you work with on the top of your assignment. If you worked alone, write that you worked alone.
        \item Show your work.
        \item Include all code you use as copyable \verb|monospaced| text in the PDF (i.e. not as a screenshot).
        \item Do not put the solutions to multiple problems on the same page.
        \item Tag your responses on gradescope. Each page should have a \emph{single} problem tag. Improperly tagged responses will not receive credit.
\end{itemize}
    
\vspace{2em}

\begin{problem}
    Consider the problem of evaluating the function $f:\mathbb{R}\to\mathbb{R}$ on some domain $D \subset \mathbb{R}$.
    We can formalize our definition of condition number of $f$ at an input $x$ as
    \begin{equation*}
        \kappa(f,x) = \lim_{\tilde{x} \to x} \frac{|f(\tilde{x}) - f(x)|}{|\tilde{x}-x|}.
    \end{equation*}

    \begin{enumerate}
        \item Suppose $f(x)$ is differentiable at $x$. 
            How does $\kappa(f,x)$ relate to $f'(x)$?
        \item Let $f(x) = \sqrt{x}$.
            What is $\kappa(f,4)$? Show $\kappa(f,0) = \infty$.
        \item Suppose $g(x) = 1000 f(x)$. How do $\kappa(f,x)$ and $\kappa(g,x)$ relate?
        \item Suppose $g(x) = f(x)+1000$. How do $\kappa(f,x)$ and $\kappa(g,x)$ relate?
        \item Suppose $g(x) = f(x-1000)$. How do $\kappa(f,x)$ and $\kappa(g,x+1000)$ relate?
    \end{enumerate}
\end{problem}

\begin{problem}
We can also define a ``relative'' condition number
    \begin{equation*}
        \kappa_{\textup{rel}}(f,x) = \lim_{\tilde{x} \to x} \frac{|f(\tilde{x}) - f(x)|/|f(x)|}{|\tilde{x}-x| / |x|}.
    \end{equation*}

    \begin{enumerate}
    \item Suppose $g(x) = 1000 f(x)$. How do $\kappa_{\textup{rel}}(f,x)$ and $\kappa_{\textup{rel}}(g,x)$ relate?
    \item Suppose $g(x) = f(x)+1000$. How do $\kappa_{\textup{rel}}(f,x)$ and $\kappa_{\textup{rel}}(g,x)$ relate?
    \item Suppose $g(x) = f(x-1000)$. How do $\kappa_{\textup{rel}}(f,x)$ and $\kappa_{\textup{rel}}(g,x+1000)$ relate?
    \item Think about the implications of your responses to these problems, and then describe a real-life situation in which it makes more sense to use the relative condition number than the regular condition number.

    \end{enumerate}
\end{problem}

\clearpage
\begin{problem}
Describe whether the following problems/tasks are well-conditioned or not. 
    Justify each of your responses.

    \begin{enumerate}
         \item Problem/task: You are given a vector $[x_1, x_2]$ and must compute the solution $[z_1, z_2]$ to the linear system of equations
            \begin{equation*}
                 2.3z_1 − 1.01z_2 = x_1 
                 ,\qquad 
                 2.31 z_1 − 1.00 z_2= x_2.
            \end{equation*}
            
            Example inputs/outputs:

            \begin{center}
            \begin{tabular}{>{\centering\arraybackslash}m{2in}>{\centering\arraybackslash}m{2in}}
            \toprule
                input $x$ & solution $f(x)$ \\ \midrule
                $[1,2]$ & $[30.816,69.184]$ \\
                $[1,1]$ & $[0.30211,-0.302115]$ \\ 
                \bottomrule
            \end{tabular}
            \end{center}

        \item Problem/task: You are given function $h:[-1,1]\to \mathbb{R}$ and must return $\int_{-1}^{1} g(s) \d{s}$. Here you should interpret $\tilde{h} \approx h$ to mean $\tilde{h}(s) \approx h(s)$ for all $s\in[-1,1]$.


            Example inputs/outputs:

            \begin{center}
            \begin{tabular}{>{\centering\arraybackslash}m{2in}>{\centering\arraybackslash}m{2in}}
            \toprule
                input $x$ & solution $f(x)$ \\ \midrule
                $h(s) = 1$ & 2 \\
                $h(s) = s^2$ & 2/3 \\
                $h(s) = \sin(s)$ & 0 \\
                \bottomrule
            \end{tabular}
            \end{center}

\clearpage


        \item Problem/task: You're designing a self-driving car. 
            As you approach an intersection, you are given a color image of the stoplight. 
            You must determine whether it is green or red; i.e. if the car should continue through the intersection or stop.
            
            Example inputs/outputs:

            \begin{center}
            \begin{tabular}{>{\centering\arraybackslash}m{2in}>{\centering\arraybackslash}m{2in}}
            \toprule
                input $x$ & solution $f(x)$ \\ \midrule
            \includegraphics[width=1.75in]{green_day.jpg} & green \\
            \includegraphics[width=1.75in]{red_day.jpg} & red \\
            \includegraphics[width=1.75in]{green_night.jpg} & green \\
            \includegraphics[width=1.75in]{red_night.jpg} & red \\
                \bottomrule
            \end{tabular}
            \end{center}

            \clearpage
        \item Problem/task: You're designing a self-driving car. 
            As you approach an intersection, you are given a greyscale image of the stoplight. 
            You must determine whether it is green or red; i.e. if the car should continue through the intersection or stop.
        
            Example inputs/outputs:

            \begin{center}
            \begin{tabular}{>{\centering\arraybackslash}m{2in}>{\centering\arraybackslash}m{2in}}
            \toprule
                input $x$ & solution $f(x)$ \\ \midrule
            \includegraphics[width=1.75in]{green_day_gs.jpg} & green \\
            \includegraphics[width=1.75in]{red_day_gs.jpg} & red \\
            \includegraphics[width=1.75in]{green_night_gs.jpg} & green \\
            \includegraphics[width=1.75in]{red_night_gs.jpg} & red \\
                \bottomrule
            \end{tabular}
            \end{center}

            Images from:
            \href{https://www.youtube.com/watch?v=yhCAWZlv_Sc}{Brooklyn 4K - Night Drive}
            and \href{https://www.youtube.com/watch?v=eyu0d1x5TMs}{Jazz New York City Drive}

    \end{enumerate}
\end{problem}
\clearpage

\begin{problem}
Consider the problem of adding $n$ numbers. 
That is $f(x_1, \ldots, x_n) = x_1 + x_2 + \cdots + x_n$. 

There are many ways we could implement an algorithm for this problem. Perhaps the simplest is
\begin{lstlisting}
import numpy as np

def add_list(x,n):
    s = np.zeros(1,dtype=x.dtype)
    for i in range(n):
        s = s + x[i]
    return s
\end{lstlisting}

Consider $x = [10^{8},1,1,\ldots, 1]$ for $n=34412$.
We can set this problem up as
\begin{lstlisting}
n = 34412
x  = np.ones(n,dtype=np.float32)
x[0] = 1*10**8

add_list(x,n)
\end{lstlisting}

Note here we are using 32 bit floating point arithmetic throughout. 

    \begin{enumerate}
        \item
            What is the correct answer to this problem, and what does the algorithm output?

        \item Suppose instead we implement an algorithm
\begin{lstlisting}
def add_list_sorted(x,n):
    z = np.sort(x)
    s = np.array(1,dtype=x.dtype)
    for i in range(n):
        s = s + z[i]
    return s
\end{lstlisting}
    What is the output of \lstinline{add_list_sorted(x,n)}?

\item Explain why the two algorithms have different outputs.
    \end{enumerate}


\clearpage
\begin{problem}

    In class (and the notes) we saw that our impelemntation of a triangular solved was faster than numpy's \lstinline{np.linalg.solve}.

    Scipy has a solver \href{https://docs.scipy.org/doc/scipy/reference/generated/scipy.linalg.solve_triangular.html}{\lstinline{sp.linalg.solve_triangular}} for triangular systems which uses LAPACK's triangular solver.

    Add this solver to the runtime comparison of \lstinline{np.linalg.solve} and our triangular solver in Chapter 7 of the notes (code is online in \href{https://drive.google.com/drive/u/0/folders/17mCJ0l6Sc4sKMXJPEklNQuDfrS6X4jcC}{Google Drive}).
    
    Include the new figure showing the runtime scaling of all three algorithms.
    How does scipy's triangular solver compare compare?

    Remember to include your code as well as the figure.

    Note also that the timings don't work that well in colab, so it is preferable to run locally if possible.

\end{problem}
\end{problem}

\end{document}

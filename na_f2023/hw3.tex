\documentclass[12pt]{article}

\usepackage[margin=1.25in]{geometry}

\usepackage{graphicx}
\usepackage{amsmath,amsthm,amssymb}

% you will probably have to remove the font-setup
\usepackage[no-math]{fontspec}
\setmainfont[
    BoldFont = Vollkorn Bold,
    ItalicFont = Vollkorn Italic,
    BoldItalicFont={Vollkorn Bold Italic},
    RawFeature=+lnum,
]{Vollkorn}

\setsansfont[
    BoldFont = Lato Bold,
    FontFace={l}{n}{*-Light},
    FontFace={l}{it}{*-Light Italic},
]{Lato}

\usepackage{unicode-math}
\mathitalicsmode=1

\setmathfont[
mathit = sym,
mathup = sym,
mathbf = sym,
math-style = TeX, 
bold-style = TeX
]{Wholegrain Math}
\everydisplay{\Umathoperatorsize\displaystyle=4ex}
\AtBeginDocument{\renewcommand\setminus{\smallsetminus}}
% font setup ends here

\newcommand\extrafootertext[1]{%
    \bgroup
    \renewcommand\thefootnote{\fnsymbol{footnote}}%
    \renewcommand\thempfootnote{\fnsymbol{mpfootnote}}%
    \footnotetext[0]{#1}%
    \egroup
}

\theoremstyle{definition}
\newtheorem{problem}{Problem}
\newenvironment{solution}
  {\textbf{Solution.} }
  {}

\usepackage{listings}
\lstset{basicstyle=\ttfamily,breaklines=true}

\usepackage{enumitem}
\setlist[itemize]{itemsep=-.15em, topsep=-.5em}
\setlist[enumerate]{itemsep=-.15em, topsep=-.5em}
\setlist[enumerate,1]{label={(\alph*)}}

\PassOptionsToPackage{hyphens}{url}
\usepackage{hyperref}

\setlength{\parskip}{1em}
\setlength{\parindent}{0em}

\renewcommand{\d}{\mathrm{d}}
\renewcommand{\vec}{\mathbf}
\newcommand{\T}{\mathsf{T}}
\newcommand{\F}{\mathsf{F}}

\usepackage{dsfont}
\newcommand{\bOne}{\mathds{1}}

\usepackage{array,booktabs}

\begin{document}

\textbf{\Large\sffamily{Homework 3\hfill Numerical Analysis Fall 2023}}
    
    \vspace{-1.8em}
    \hrulefill
 
\textbf{Instructions:}
    \begin{itemize}
        \item Due 10/13 at 5:00pm on \href{https://www.gradescope.com/courses/570477/}{Gradescope}.
        \item You must follow the submission policy in the \href{https://courses.chen.pw/na_f2023/syllabus.html}{syllabus} 
\end{itemize}
   
\vspace{.5em}


\begin{problem}
    For each problem, write down a matrix or matrices which performs the stated operations to a $3\times 4$ matrix.
    Make sure to specify whether you should be applied on the left or right.

    For reference, we will also show what the operation does to the following matrices:
    \[
        \vec{A} = 
        \begin{bmatrix}
            1 & 2 & 3 & 4 \\
            5 & 6 & 7 & 8 \\
            9 & 10 & 11 & 12
        \end{bmatrix}
        ,\qquad 
        \vec{B} = 
        \begin{bmatrix}
            1 & 3& -2 & 0 \\
            2 & -3 & 1 & 1 \\
            2 & -2 & 5 & 4
        \end{bmatrix}
    \]

    \begin{enumerate}
        \item Extract the second column.
            \[
                \vec{A} \to \begin{bmatrix} 2 \\ 6 \\ 10 \end{bmatrix}
                    ,\qquad
                    \vec{B} \to \begin{bmatrix} 3 \\ -3 \\ -2 \end{bmatrix}
            \]
        \item Extract the second column and place it in the third column of a $3\times 3$ matrix of zeros.
            \[
                    \vec{A} \to \begin{bmatrix} 0&0&2 \\ 0&0&6 \\ 0&0&10 \end{bmatrix}
                    ,\qquad
                    \vec{B} \to \begin{bmatrix} 0&0&3 \\ 0&0&-3 \\ 0&0&-2 \end{bmatrix}
            \]
        \item Swap the second and third columns.
    \[
        \vec{A} \to 
        \begin{bmatrix}
            1 & 3 & 2 & 4 \\
            5 & 7 & 6 & 8 \\
            9 & 11 & 10 & 12
        \end{bmatrix}
        ,\qquad 
        \vec{B} \to
        \begin{bmatrix}
            1 & -2 & 3& 0 \\
            2 & 1 & -3 & 1 \\
            2 & 5 & -2 & 4
        \end{bmatrix}
    \]
    \item Sum up each column.
    \[
        \vec{A} \to  
        \begin{bmatrix}
            15 & 18 & 21 & 24 
        \end{bmatrix}
        ,\qquad 
        \vec{B} \to 
        \begin{bmatrix}
            5 & -2 & 4 & 5
        \end{bmatrix}
    \]
    \item Swap the second and third columns, then sum up each column  (this one requires using two matrices).
    \[
        \vec{A} \to  
        \begin{bmatrix}
            15 & 21 & 18 & 24 
        \end{bmatrix}
        ,\qquad 
        \vec{B} \to 
        \begin{bmatrix}
            5 & 4 & -2 & 5
        \end{bmatrix}
    \]

    \item Sum all the entries (this one requires using two matrices).
    \[
        \vec{A} \to  
        \begin{bmatrix}
            78
        \end{bmatrix}
        ,\qquad 
        \vec{B} \to 
        \begin{bmatrix}
            12
        \end{bmatrix}
    \]

    \end{enumerate}
\end{problem}


\begin{problem}

    \begin{enumerate}
        \item Prove that on a log-log plot, $n$ vs $n^k$ is a line. What is the slope?
        \item What will the plot of $n$ vs $5n^2 + 3n + 1$ look like on a log-log plot when $x$ is large?
        \item Prove that on a log-y plot, $n$ vs $\rho^n$ is a line. What is the slope?
    \end{enumerate}

\end{problem}

\begin{problem}

    In chapter 9, we see our implementation of a triangular solved was faster than numpy's \lstinline{np.linalg.solve}.

    Scipy has a solver \href{https://docs.scipy.org/doc/scipy/reference/generated/scipy.linalg.solve_triangular.html}{\lstinline{sp.linalg.solve_triangular}} for triangular systems which uses LAPACK's triangular solver.

    Add this solver to the runtime comparison of \lstinline{np.linalg.solve} and our triangular solver in Chapter 9 of the notes (all the code from the notes is online on the Google Drive).
    
    Include the new figure showing the runtime scaling of all three algorithms.
    How does scipy's triangular solver compare compare?

    Remember to include your code as well as the figure.

    Note also that the timings don't work that well in colab, so it is preferable to run locally if possible.

\end{problem}

\begin{problem}
Consider the problem of adding $n$ numbers. 
That is $f(x_1, \ldots, x_n) = x_1 + x_2 + \cdots + x_n$. 

There are many ways we could implement an algorithm for this problem. Perhaps the simplest is
\begin{lstlisting}
import numpy as np

def add_list(x,n):
    s = np.zeros(1,dtype=x.dtype)
    for i in range(n):
        s = s + x[i]
    return s
\end{lstlisting}

Consider $x = [10^{8},1,1,\ldots, 1]$ for $n=34412$.
We can set this problem up as
\begin{lstlisting}
n = 34412
x  = np.ones(n,dtype=np.float32)
x[0] = 1*10**8

add_list(x,n)
\end{lstlisting}

Note here we are using 32 bit floating point arithmetic throughout. 

    \begin{enumerate}
        \item
            What is the correct answer to this problem, and what does the algorithm output?

        \item Suppose instead we implement an algorithm
\begin{lstlisting}
def add_list_sorted(x,n):
    z = np.sort(x)
    s = np.zeros(1,dtype=x.dtype)
    for i in range(n):
        s = s + z[i]
    return s
\end{lstlisting}
    What is the output of \lstinline{add_list_sorted(x,n)}?

\item Give a possible explication for why the two algorithms have different outputs.
    \end{enumerate}

\end{problem}


\clearpage
\begin{problem}
Describe whether the following problems/tasks are well-conditioned or not. 
    Justify each of your responses.

    \begin{enumerate}
         \item Problem/task: You are given a vector $[x_1, x_2]$ and must compute the solution $[z_1, z_2]$ to the linear system of equations
            \begin{equation*}
                 2.3z_1 − 1.01z_2 = x_1 
                 ,\qquad 
                 2.31 z_1 − 1.00 z_2= x_2.
            \end{equation*}
            
            Example inputs/outputs:

            \begin{center}
            \begin{tabular}{>{\centering\arraybackslash}m{2in}>{\centering\arraybackslash}m{2in}}
            \toprule
                input $x$ & solution $f(x)$ \\ \midrule
                $[1,2]$ & $[30.816,69.184]$ \\
                $[1,1]$ & $[0.30211,-0.302115]$ \\ 
                \bottomrule
            \end{tabular}
            \end{center}

        \item Problem/task: You are given function $h:[-1,1]\to \mathbb{R}$ and must return $\int_{-1}^{1} g(s) \d{s}$. 
            We will use the norm $\|\tilde{h} - h\|_\infty := \max_{s\in[-1,1]} |\tilde{h}(x) - h(x)|$.

            Example inputs/outputs:

            \begin{center}
            \begin{tabular}{>{\centering\arraybackslash}m{2in}>{\centering\arraybackslash}m{2in}}
            \toprule
                input $x$ & solution $f(x)$ \\ \midrule
                $h(s) = 1$ & 2 \\
                $h(s) = s^2$ & 2/3 \\
                $h(s) = \sin(s)$ & 0 \\
                \bottomrule
            \end{tabular}
            \end{center}

\clearpage

\item Problem/task: You're designing a self-driving car. 
            As you approach an intersection, you are given a greyscale image of the stoplight. 
            You must determine whether it is green or red; i.e. if the car should continue through the intersection or stop.
        
            Example inputs/outputs:

            \begin{center}
            \begin{tabular}{>{\centering\arraybackslash}m{2in}>{\centering\arraybackslash}m{2in}}
            \toprule
                input $x$ & solution $f(x)$ \\ \midrule
            \includegraphics[width=1.75in]{green_day_gs.jpg} & green \\
            \includegraphics[width=1.75in]{red_day_gs.jpg} & red \\
            \includegraphics[width=1.75in]{green_night_gs.jpg} & green \\
            \includegraphics[width=1.75in]{red_night_gs.jpg} & red \\
                \bottomrule
            \end{tabular}
            \end{center}
\clearpage


        \item Problem/task: You're designing a self-driving car. 
            As you approach an intersection, you are given a color image of the stoplight. 
            You must determine whether it is green or red; i.e. if the car should continue through the intersection or stop.
            
            Example inputs/outputs:

            \begin{center}
            \begin{tabular}{>{\centering\arraybackslash}m{2in}>{\centering\arraybackslash}m{2in}}
            \toprule
                input $x$ & solution $f(x)$ \\ \midrule
            \includegraphics[width=1.75in]{green_day.jpg} & green \\
            \includegraphics[width=1.75in]{red_day.jpg} & red \\
            \includegraphics[width=1.75in]{green_night.jpg} & green \\
            \includegraphics[width=1.75in]{red_night.jpg} & red \\
                \bottomrule
            \end{tabular}
            \end{center}

       
            Images from:
            \href{https://www.youtube.com/watch?v=yhCAWZlv_Sc}{Brooklyn 4K - Night Drive}
            and \href{https://www.youtube.com/watch?v=eyu0d1x5TMs}{Jazz New York City Drive}

    \end{enumerate}
\end{problem}

\end{document}

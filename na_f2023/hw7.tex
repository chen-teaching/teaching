\documentclass[12pt]{article}

\usepackage[margin=1.25in]{geometry}

\usepackage{graphicx}
\usepackage{amsmath,amsthm,amssymb}

% you will probably have to remove the font-setup
\usepackage[no-math]{fontspec}
\setmainfont[
    BoldFont = Vollkorn Bold,
    ItalicFont = Vollkorn Italic,
    BoldItalicFont={Vollkorn Bold Italic},
    RawFeature=+lnum,
]{Vollkorn}

\setsansfont[
    BoldFont = Lato Bold,
    FontFace={l}{n}{*-Light},
    FontFace={l}{it}{*-Light Italic},
]{Lato}

\usepackage{unicode-math}
\mathitalicsmode=1

\setmathfont[
mathit = sym,
mathup = sym,
mathbf = sym,
math-style = TeX, 
bold-style = TeX
]{Wholegrain Math}
\everydisplay{\Umathoperatorsize\displaystyle=4ex}
\AtBeginDocument{\renewcommand\setminus{\smallsetminus}}
% font setup ends here

\newcommand\extrafootertext[1]{%
    \bgroup
    \renewcommand\thefootnote{\fnsymbol{footnote}}%
    \renewcommand\thempfootnote{\fnsymbol{mpfootnote}}%
    \footnotetext[0]{#1}%
    \egroup
}

\theoremstyle{definition}
\newtheorem{problem}{Problem}
\newenvironment{solution}
  {\textbf{Solution.} }
  {}

\usepackage{listings}
\lstset{basicstyle=\ttfamily,breaklines=true}

\usepackage{enumitem}
\setlist[itemize]{itemsep=-.15em, topsep=-.5em}
\setlist[enumerate]{itemsep=-.15em, topsep=-.5em}
\setlist[enumerate,1]{label={(\alph*)}}

\PassOptionsToPackage{hyphens}{url}
\usepackage{hyperref}

\setlength{\parskip}{1em}
\setlength{\parindent}{0em}

\renewcommand{\d}{\mathrm{d}}
\renewcommand{\vec}{\mathbf}
\newcommand{\T}{\mathsf{T}}
\newcommand{\F}{\mathsf{F}}

\usepackage{dsfont}
\newcommand{\bOne}{\mathds{1}}

\usepackage{array,booktabs}

\begin{document}

\textbf{\Large\sffamily{Homework 7\hfill Numerical Analysis Fall 2023}}
    
    \vspace{-1.8em}
    \hrulefill
 
\textbf{Instructions:}
    \begin{itemize}
        \item Due 12/15 at 5:00pm on \href{https://www.gradescope.com/courses/570477/}{Gradescope}.
        \item You must follow the submission policy in the \href{https://courses.chen.pw/na_f2023/syllabus.html}{syllabus} 
        \item This homework is worth 40 points (less than the usual 50 points)
        \item If this is your lowest homework grade, I will drop it (i.e. this homework is optional)
\end{itemize}
   
\vspace{.5em}


\begin{problem}
    Spend at least two hours working on your project prior to the 20th. 
    Answer the following:
    \begin{enumerate}
        \item What is the current status of your project? 
        \item What are the big tasks you have left to do before your project is done?
        \item What is your plan for completing the project in a timely manner?
    \end{enumerate}
\end{problem}



\begin{problem}
    Suppose $\vec{A}$ has SVD $\vec{A} = \vec{U} \vec{\Sigma} \vec{V}^\T$ where 
    \[
        \vec{U} = 
        \begin{bmatrix}
            |&|&&|\\
            \vec{u}_1 & \vec{u}_2 & \cdots & \vec{u}_n\\
            |&|&&|\\
        \end{bmatrix},
        \qquad
        \vec{\Sigma} = \begin{bmatrix} \sigma_1 \\ &\sigma_2 \\ &&\ddots \\ &&& \sigma_n\end{bmatrix}
        ,\qquad
        \vec{V}
    =   \begin{bmatrix}
            |&|&&|\\
            \vec{v}_1 & \vec{v}_2 & \cdots & \vec{v}_n\\
            |&|&&|\\
        \end{bmatrix}. 
    \]
    \begin{enumerate}
        \item Show that $\vec{v}_i$ is an eigenvector of $\vec{A}^\T \vec{A}$. What is the corresponding eigenvalue?

        \item

    Define the block matrix:
    \[
        \vec{B} = 
        \begin{bmatrix} \vec{0}& \vec{A} \\ 
            \vec{A}^\T & \vec{0}
        \end{bmatrix}.
    \]
    Show that 
    \[
        \vec{x} = \begin{bmatrix} \vec{u}_i \\ \vec{v}_i \end{bmatrix}
    \]
    is an eigenvector of $\vec{B}$.
    What is the corresponding eigenvalue?
    \end{enumerate}
\end{problem}


\begin{problem}
    Let $f(x) = |x|$ and define $p_k(x)$ as the degree $k$ polynomial interpolate to $f(x)$ at the $k+1$ Chebyshev nodes. 
    \begin{enumerate}
        \item 
            For $k=10$ make one plot with $x$ vs $f(x)$ and $x$ vs $p_k(x)$  and one plot over $x$ vs $|f(x) - p_k(x)|$ with the vertical axis on a log-scale.

        \item 
            Repeat this for $k=40$


    \end{enumerate}
\end{problem}


\begin{problem}
    \begin{enumerate}
        \item 
    Let $f(x) = \exp(-x)$ and define $p_k(x)$ as the degree $k$ polynomial interpolate to $f(x)$ at the $k+1$ Chebyshev nodes. 

    Make a plot of $k$ vs 
    \[
        \max_{x\in[-1,1]} |f(x) - p_k(x)|
    \]
    for $k=0,1,2,\ldots20$. Put the $y$-axis on a log scale and label the axes/plot/etc.

    To approximate \[
        \max_{x\in[-1,1]} |f(x) - p_k(x)|
    \]
    you can instead take the maximum over $1000$ equally spaced points in $[-1,1]$ and use this instead. 

\item Repeat this for $f(x) = 1/(1+16x^2)$ and $k=0,1,\ldots, 100$.

\item Repeat this for $f(x) = |\sin(5x)|^3 = (\sin(5x)^2)^{3/2}$ and $k=0,1,\ldots,100$, but put both axes on log-scales.

    Add a line $k$ vs $k^{-\nu}$, where $\nu$ is the largest value so that the $(\nu-1)$-st derivative of $f(x)$ is continuous.
\end{enumerate}

\end{problem}

\begin{problem}
    Let $f(x) = 1/(1+16x^2)$.

    For any non-negative integer $k$, set $n=k^2+1$ and let $x_1 ,\ldots, x_n$ be $n$ equally spaced points from $-1$ to $1$ and let $q_{k}(x)$ be the degree $k$ polynomial minimizing
    \[
        \min_{\deg(q) = k} \sum_{i=1}^{n} ( f(x_i) - q(x_i) )^2.
    \]

    On a log-y plot, plot the error
    \[
        \max_{x\in[-1,1]} |f(x) - q_{k}(x)|
    \]
    for $k=0,1,\ldots, 100$.
    Add to this plot the error of the Chebyshev interpolant that you computed in 3(b).



\end{problem}


\end{document}

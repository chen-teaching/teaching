\documentclass[12pt]{article}

\usepackage[margin=1.25in]{geometry}

\usepackage{graphicx}
\usepackage{amsmath,amsthm,amssymb}

% you will probably have to remove the font-setup
\usepackage[no-math]{fontspec}
\setmainfont[
    BoldFont = Vollkorn Bold,
    ItalicFont = Vollkorn Italic,
    BoldItalicFont={Vollkorn Bold Italic},
    RawFeature=+lnum,
]{Vollkorn}

\setsansfont[
    BoldFont = Lato Bold,
    FontFace={l}{n}{*-Light},
    FontFace={l}{it}{*-Light Italic},
]{Lato}

\usepackage{unicode-math}
\mathitalicsmode=1

\setmathfont[
mathit = sym,
mathup = sym,
mathbf = sym,
math-style = TeX, 
bold-style = TeX
]{Wholegrain Math}
\everydisplay{\Umathoperatorsize\displaystyle=4ex}
\AtBeginDocument{\renewcommand\setminus{\smallsetminus}}
% font setup ends here

\newcommand\extrafootertext[1]{%
    \bgroup
    \renewcommand\thefootnote{\fnsymbol{footnote}}%
    \renewcommand\thempfootnote{\fnsymbol{mpfootnote}}%
    \footnotetext[0]{#1}%
    \egroup
}

\theoremstyle{definition}
\newtheorem{problem}{Problem}
\newenvironment{solution}
  {\textbf{Solution.} }
  {}

\usepackage{listings}
\lstset{basicstyle=\ttfamily,breaklines=true}

\usepackage{enumitem}
\setlist[itemize]{itemsep=-.15em, topsep=-.5em}
\setlist[enumerate]{itemsep=-.15em, topsep=-.5em}
\setlist[enumerate,1]{label={(\alph*)}}

\PassOptionsToPackage{hyphens}{url}
\usepackage{hyperref}

\setlength{\parskip}{1em}
\setlength{\parindent}{0em}

\renewcommand{\d}{\mathrm{d}}
\renewcommand{\vec}{\mathbf}
\newcommand{\T}{\mathsf{T}}
\newcommand{\F}{\mathsf{F}}

\usepackage{dsfont}
\newcommand{\bOne}{\mathds{1}}

\usepackage{array,booktabs}

\begin{document}


\textbf{\Large\sffamily{Homework 6\hfill Numerical Analysis Fall 2024}}
    
    \vspace{-1.8em}
    \hrulefill

    \textbf{Instructions:}
    \begin{itemize}
        \item Due 11/25 at 6:00pm on \href{https://www.gradescope.com/courses/818054}{Gradescope}.
        \item You must follow the submission policy in the \href{https://courses.chen.pw/na_f2024/syllabus.html}{syllabus} 
\end{itemize}
   
\vspace{.5em}

\begin{problem}
    Suppose we have time-series data $(t_1, y_1), \ldots, (t_n,y_n)$. 
    We can try to fit the data with a polynomial of degree $k$. 
    I.e. find a polynomial 
    \[
    p(x) = c_0 + c_1 x + \cdots + c_k x^k
    \]
    so that at each time $t_i$, we have
    \[
    p(t_i) \approx y_i.
    \]
    To do this, we can solve a least squares problem
    \[
        \min_{c_0, \ldots, c_k} \sum_{i=1}^{n} ( y_i - p(t_i))^2
        =
        \min_{c_0, \ldots, c_k} \sum_{i=1}^{n} ( y_i - (c_0 + c_1 t_i + \cdot + c_k t_i^k))^2.
    \]

    As we saw in class, this can be written as a linear algebra problem:
    \[
    \min_{\vec{c}\in\mathbb{R}^{k+1}} \| \vec{b} - \vec{A}\vec{x} \|_2^2
    \]
    where
    \[
    \vec{b} = \begin{bmatrix} y_1 \\ y_2 \\ \vdots \\ y_n \end{bmatrix}
    ,\qquad
    \vec{A} = 
    \begin{bmatrix} 
        1 & t_1 & t_1^2 & \cdots & t_1^k \\
        1 & t_2 & t_2^2 & \cdots & t_2^k \\
        \vdots & \vdots &&\vdots \\
        1 & t_n & t_n^2 & \cdots & t_n^k 
    \end{bmatrix}.
    \]

    Get the data \texttt{temp.npy} from the website. 
    Load the data:    
    \begin{lstlisting}
temp = np.load('gdrive/MyDrive/na_f2024/hw files/temp.npy')
t = np.arange(190)/24 # time in days
    \end{lstlisting}

    We can plot the data:
    \begin{lstlisting}
plt.subplots(1,1,figsize=(12,4))
plt.plot(t,temp,marker='.',ls='None',label='data')
plt.ylabel('temperature (F)')
plt.xlabel('time since Oct 29 (days)')
plt.legend()
    \end{lstlisting}


    \begin{enumerate}
        \item 
        
        For each $k=5, 10, 15$, set up the least squares problem for and and solve it (either using \texttt{np.linalg.lstsq} or using a QR factorization followed by a triangular solve).

        Add each of the polynomials (evaluated at a finer grid of $t$ values) to the plot. Make sure they are labeled. 
    
        \item Note that we could represent our polynomial in terms of a different basis. I.e. instead of $1, x, x^2, \ldots, x^k$, we could use any family $p_0(x), p_1(x), \ldots, p_k(x)$, where $p_i(x)$ has degree $i$. 
        
        One common choice is the Chebyshev polynomials. On $[-1,1]$, these are  defined by
        \[
        T_{0}(x) = 1, \quad T_{1}(x) = x, \quad T_{j+1}(x) = 2xT_j(x) - T_j{-1}(x).
        \]
        More generally, we can define them on an interval $[a,b]$ by 
        \[
        p_i(x) = T_i\left(\frac{2x - (a+b)}{b-a}\right).
        \]

        Repeat the above process with the scaled Chebyshev polynomials (here $a=0$ and $b=8$); i.e. using 
        \[
        \vec{A} = 
        \begin{bmatrix}
        p_0(t_1) & p_1(t_1) & \cdots & p_k(t_1) \\
        T_0(t_2) & p_1(t_2) & \cdots & p_k(t_2) \\
        \vdots & \vdots & & \vdots \\
        p_0(t_n) & p_1(t_n) & \cdots & p_k(t_n) \\
        \end{bmatrix}.
        \]

        We can make a function to evaluate the Chebyshev polynomials on $[a,b]$ as follows: \footnote{It's not obvious how to get this formula, but you could prove it satisfies the recurrence formula! You can look at the wikipedia page for more info}
        \begin{lstlisting}
def chebyshev_polynomail(j,x,a,b):
    return np.cos(j*np.arccos((2*x-(a+b))/(b-a)))
                    
        \end{lstlisting}

        If we want to evaluate this for $j=3$ at all the $t$ values we can do:
        \begin{lstlisting}
chebyshev_polynomail(3,t,0,8)
        \end{lstlisting} 

        Make a plot with $k=5,10,15$.

        \item Explain why the plots should look the same if we were doing th e computations exactly.
        
        \item Look at the condition numbers of all of the matrices you use in the least squares problems. 
        How does this explain why the plots with different polynomial families look different?

    \end{enumerate}

\end{problem}


\begin{problem}

    Suppose $\vec{A}$ is symmetric with eigenvalues $\lambda_1, \ldots, \lambda_n$ (so that $|\lambda_1| > |\lambda_2|  \geq \cdots \geq |\lambda_n|$) and corresponding orthonormal eigenvectors $\vec{v}_1, \ldots, \vec{v}_n$.

    Our analysis of the power-method involved writing the starting vector $\vec{x}$ in terms of $\vec{V}$; i.e. writing 
    \[
    \vec{x} = \vec{V} \begin{bmatrix}
        c_1 \\ c_2 \\ \vdots \\ c_n
    \end{bmatrix}.
    \]
    As long as $|c_1| > 0$, then we got convergence to $\vec{v}_1$.

    Of course, in practice we don't know the eigenvectors or the $c_i$s. 
    However, it turns out if we choose $\vec{x}$ randomly, then $c_1$ will never be zero (and in fact will never be that small).

    \begin{enumerate}
        \item Suppose we have $\vec{x}$ and $\vec{V}$. 
        How we compute $c_1$?
        
        \item Make any $5\times 5$ symmetric matrix $\vec{A}$ sample a length 5 vector $\vec{x}$ whose entries are independent Gaussians. You can do this by using \texttt{np.random.randn(5)}.
        
        \item Use numpy's \texttt{np.linalg.eigh} to compute its eigendecompsition. Use (a) to compute $c_1$ and report it's value.
        
        \item Repeat (b) 1000 times with a new $\vec{x}$ each time. 
        Make a histogram of the value of $c_1$ over these trials.\footnote{It turns out the distribution of $c_1$ does not depend on how you generated the matrix $\vec{A}$! This is because of something called the orthogonal invariance of the Gaussian distribution.}
 
        Was $c_1$ ever zero? 
        \item In the above process, we needed to compute $\vec{V}$ in order to find $c_1$. But this would be expensive for large matrices.
        
        Explain why we do not need to compute $\vec{V}$ in practice in order to use the power method to find $\vec{v}_1$. 
        
    \end{enumerate}


\end{problem}



\begin{problem}
Suppose $\vec{A}$ is symmetric with eigenvalue decomposition $\vec{V}\vec{\Lambda}\vec{V}^\T$, where $\vec{\Lambda}$ is diagonal with entries $\lambda_1, \ldots, \lambda_n$.

\begin{enumerate}
    \item 
Find the eigenvalue decomposition of:
\begin{itemize}
\item $\vec{A}^k$    
\item $\vec{A}^3 -2 \vec{A}$
\item $\vec{A}^{-1}$
\item $(\vec{A}+\lambda)^{-1}$
\end{itemize}
\item What is the largest eigenvalue of $(\vec{A}+\lambda)^{-1}$ in terms of the $\lambda_i$s?
\end{enumerate}
\end{problem}


\begin{problem}
    Suppose $\vec{A}$ is symmetric with eigenvalues $\lambda_1, \ldots, \lambda_n$ (so that $|\lambda_1| > |\lambda_2| > |\lambda_3| > \cdots > |\lambda_n|$) and corresponding orthonormal eigenvectors $\vec{v}_1, \ldots, \vec{v}_n$.

    Define $\vec{B} = \vec{A}(\vec{I} - \vec{v}_1\vec{v}_1^\T)$. 

    \begin{enumerate}
        \item What are the eigenvalues of $\vec{B}$?

        \item Explain how to use the observation in (a) and the power-method to find $\vec{v}_2$.

    \end{enumerate}

\end{problem}


\end{document}

\documentclass[12pt]{article}

\usepackage[margin=1.25in]{geometry}

\usepackage{graphicx}
\usepackage{amsmath,amsthm,amssymb}

% you will probably have to remove the font-setup
\usepackage[no-math]{fontspec}
\setmainfont[
    BoldFont = Vollkorn Bold,
    ItalicFont = Vollkorn Italic,
    BoldItalicFont={Vollkorn Bold Italic},
    RawFeature=+lnum,
]{Vollkorn}

\setsansfont[
    BoldFont = Lato Bold,
    FontFace={l}{n}{*-Light},
    FontFace={l}{it}{*-Light Italic},
]{Lato}

\usepackage{unicode-math}
\mathitalicsmode=1

\setmathfont[
mathit = sym,
mathup = sym,
mathbf = sym,
math-style = TeX, 
bold-style = TeX
]{Wholegrain Math}
\everydisplay{\Umathoperatorsize\displaystyle=4ex}
\AtBeginDocument{\renewcommand\setminus{\smallsetminus}}
% font setup ends here

\newcommand\extrafootertext[1]{%
    \bgroup
    \renewcommand\thefootnote{\fnsymbol{footnote}}%
    \renewcommand\thempfootnote{\fnsymbol{mpfootnote}}%
    \footnotetext[0]{#1}%
    \egroup
}

\theoremstyle{definition}
\newtheorem{problem}{Problem}
\newenvironment{solution}
  {\textbf{Solution.} }
  {}

\usepackage{listings}
\lstset{basicstyle=\ttfamily,breaklines=true}

\usepackage{enumitem}
\setlist[itemize]{itemsep=-.15em, topsep=-.5em}
\setlist[enumerate]{itemsep=-.15em, topsep=-.5em}
\setlist[enumerate,1]{label={(\alph*)}}

\PassOptionsToPackage{hyphens}{url}
\usepackage{hyperref}

\setlength{\parskip}{1em}
\setlength{\parindent}{0em}

\renewcommand{\d}{\mathrm{d}}
\renewcommand{\vec}{\mathbf}
\newcommand{\T}{\mathsf{T}}
\newcommand{\F}{\mathsf{F}}

\usepackage{dsfont}
\newcommand{\bOne}{\mathds{1}}


\begin{document}

\textbf{\Large\sffamily{Homework 2\hfill Numerical Analysis Fall 2024}}
    
    \vspace{-1.8em}
    \hrulefill
 
\textbf{Instructions:}
    \begin{itemize}
        \item Due 09/24 at 5:00pm on \href{https://www.gradescope.com/courses/818054}{Gradescope}.
        \item You must follow the submission policy in the \href{https://courses.chen.pw/na_f2024/syllabus.html}{syllabus} 
\end{itemize}

\vspace{5em}



\begin{problem}
Consider the matrix 
    \[
        \vec{A} 
        = 
        \begin{bmatrix}
            0 & -1 \\
            1 & 0
        \end{bmatrix}
        \begin{bmatrix}
            3 & 0 \\ 0 & -1 \\
        \end{bmatrix}
        \begin{bmatrix}
            \cos \pi/3 & -\sin \pi/3 \\
            \sin \pi/3 & \cos \pi/3
        \end{bmatrix}
    \]
    \begin{enumerate}
        \item Find an SVD $\vec{U}\vec{\Sigma}\vec{V}^\T$ of $\vec{A}$. Hint: think about why the given factorization is not an SVD.
        \item Draw what $\vec{V}^\T$ does to the following points (here a point is anything shown in black, and the dotted lines represent the x and y axes.):
            
        \includegraphics[scale=.4]{car.pdf}
        \hfill
        \includegraphics[scale=.4]{nyu.pdf}
    
        \item Draw what $\vec{\Sigma}\vec{V}^\T$ does to the points.
        \item Draw what $\vec{U}\vec{\Sigma}\vec{V}^\T$ does to the points.
    \end{enumerate}

\end{problem}

\hfill

\begin{problem}
Recall:
\[
\| \vec{A} \|_2 = \max_{\vec{z} \neq 0} \frac{\| \vec{A}\vec{z} \|_2}{\| \vec{z} \|_2}
,\qquad
\| \vec{A} \|_\F^2 = \sum_{i,j} [\vec{A}]_{i,j}^2 
\]
We will prove the identity $\|\vec{A}\vec{B}\|_\F \leq \|\vec{A}\|_2 \| \vec{B} \|_\F$.
\begin{enumerate}
    \item 
        Prove that $\| \vec{A}\vec{x} \|_2 \leq \|\vec{A}\|_2 \|\vec{x}\|_2$ for any vector $\vec{x}$.
    \item 
        Suppose $\vec{X}$ is a $n\times m$ matrix. 
        Write $\|\vec{X}\|_\F$ in terms of the column-norms $\|[\vec{X}]_{:,i}\|_2$.
    \item Use (a) and (b) to prove the identity $\|\vec{A}\vec{B}\|_\F \leq \|\vec{A}\|_2 \| \vec{B} \|_\F$.
\end{enumerate}

\end{problem}

\clearpage
\begin{problem}
    Let $\vec{A}$ be a $m\times n$ matrix with SVD $\vec{U}\vec{\Sigma}\vec{V}^\T$ where $\vec{\Sigma}$ contains the singular values $\sigma_1 \geq \sigma_2 \geq \cdots \geq \sigma_n$.
    We will prove that $\|\vec{A}\|_2 = \sigma_1$.
    \begin{enumerate}
        \item Prove that if $\vec{U}^\T\vec{U} = \vec{I}$ then $\|\vec{U}\vec{z}\|_2 = \|\vec{z}\|_2$ for any vector $\vec{z}$.
        \item Prove that if $\vec{V}^\T\vec{V} = \vec{I}$, then $\vec{V}^\T \vec{x} = \vec{0}$ if and only if $\vec{x} = \vec{0}$.
        \item Prove that $\|\vec{A}\|_2 = \|\vec{\Sigma}\|_2$.
        \item It remains to show that $\|\vec{\Sigma}\|_2 = \sigma_1$.
        \begin{enumerate}[label=(\roman*)]
            \item  Compute $\|\vec{\Sigma}\vec{x}\|_2^2$ for an arbitrary vector $\vec{x} = [x_1, x_2, \ldots, x_n]^\T$.
            \item Show that $\|\vec{\Sigma}\vec{x}\|_2^2 \leq \sigma_1^2 \| \vec{x}\|_2^2$
            \item Show that there exists a vector $\vec{x}$ such that $\|\vec{\Sigma}\vec{x}\|_2 = \sigma_1 \| \vec{x}\|_2$.
            \item Conclude that $\|\vec{\Sigma}\|_2 = \sigma_1$.
        \end{enumerate}

    \end{enumerate}
\end{problem}

\vspace{4em}

\begin{problem}

    Download the numpy data file from this link: \url{https://drive.google.com/file/d/18Xf0029XXm3ENWfe3Z0a--Cf6tjyCViz/view?usp=drive_link}

Use the following code to import the file into numpy.

\begin{lstlisting}
import numpy as np
import matplotlib.pyplot as plt

im = np.load('change this path/CIMS.npy')
\end{lstlisting}

    If you are using google colab, you can copy the \lstinline{CIMS.npy} file to your own drive and then 
\begin{lstlisting}
from google.colab import drive
drive.mount('/content/gdrive')

im = np.load('gdrive/MyDrive/change this path/CIMS.npy')
\end{lstlisting}

In both cases, forma matrix from the image data.
\begin{lstlisting}
A = np.mean(im,axis=2)
\end{lstlisting}
Here we obtain $\vec{A}$ by averaging the red, green, and blue channels of the image.
This results in a black and white image.

    \begin{enumerate}
        \item Plot the image using \lstinline{plt.imshow}.
            You may want to use the colormap \texttt{'Greys\_r'} so that it looks like a greyscale image.
        \item
            Compute the reduced SVD of $\vec{A}$. 
            You can use \lstinline{full_matrices=False} to get the reduced SVD.
            This will be much faster than computing the full SVD.

            For each $k=1,10,100,200$, make a plot of the best rank-$k$ approximation $\vec{A}_k$ to $\vec{A}$ (i.e. via truncated SVD).
            Label each plot with the rank $k$ as well as the relative error $\| \vec{A} - \vec{A}_k \|_\F / \| \vec{A} \|_\F$
        \item 
            Remark on the quality of the plots. 
            
            How many numbers are required to store $\vec{A}$?
            How many numbers are required to store the rank-$k$ truncated SVD (as a factorization)?
    \end{enumerate}

\end{problem}

\vspace{2em}
\begin{problem}
    Computing the SVD is expensive, but randomization can help us!
    \begin{enumerate}
        \item Randomized numerical linear algebra (RandNLA) is the study of the use of randomness in numerical linear algebra algorithms. 
            One of the most famous randNLA algorithms is the randomized SVD.
            A simple version for approximating the SVD of a $m\times n$ matrix $\vec{A}$ can be described in several lines:
            \begin{itemize}
                \item Choose a $n\times k$ matrix $\vec{R}$ with standard normal random entries
                \item Compute $\vec{X} = \vec{A} \vec{R}$
                \item Compute $\vec{Q},\_,\_ = \textsc{reduced-svd}(\vec{X})$
                \item 
                    $\hat{\vec{U}}, \hat{\vec{\Sigma}}, \hat{\vec{V}}^\T = \textsc{reduced-svd}(\vec{Q}^\T \vec{A})$: 
                \item Return approximate SVD of $\vec{A}$: $(\vec{Q} \hat{\vec{U}}) \hat{\vec{\Sigma}} \hat{\vec{V}}^\T$  
            \end{itemize}

            Implement this algorithm with the same matrix $\vec{A}$ as in Problem 4.
            To generate the random matrix, you can use \lstinline{np.random.randn(n,k)}.


            Again make sure to use \lstinline{full_matrices=False} when computing the SVD of $\vec{Q}^\T \vec{A}$.
            Compare this to long the whole randomized SVD took (all of the steps) with $k=100$ against the time to compute the exact SVD in the previous problem.
       
        \item Prove that the factors $\vec{Q}\hat{\vec{U}}$, $\hat{\vec{\Sigma}}$ and $\hat{\vec{V}}^\T$ have the same properties as a SVD; i.e. $\vec{Q}\hat{\vec{U}}$ and $\hat{\vec{V}}$ have orthonormal columns and $\hat{\vec{\Sigma}}$ is diagonal with non-negative entires.
        \item 
            Make a plot of the rank $k=100$ truncated SVD (from problem 4) and the $k=100$ randomized SVD.
            Show the relative errors $\|\vec{A} - (\vec{Q} \hat{\vec{U}})\hat{\vec{\Sigma}} \hat{\vec{V}}^\T \|_\F/\|\vec{A}\|_\F$ for each. 
        \item How long did this algorithm take to run vs. the reduced SVD in problem 4? Why was it so much faster? Hint: what are the dimensions of the matrices which we take the SVD of using this approach?

    \end{enumerate}
\end{problem}

\end{document}

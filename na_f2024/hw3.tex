\documentclass[12pt]{article}

\usepackage[margin=1.25in]{geometry}

\usepackage{graphicx}
\usepackage{amsmath,amsthm,amssymb}

% you will probably have to remove the font-setup
\usepackage[no-math]{fontspec}
\setmainfont[
    BoldFont = Vollkorn Bold,
    ItalicFont = Vollkorn Italic,
    BoldItalicFont={Vollkorn Bold Italic},
    RawFeature=+lnum,
]{Vollkorn}

\setsansfont[
    BoldFont = Lato Bold,
    FontFace={l}{n}{*-Light},
    FontFace={l}{it}{*-Light Italic},
]{Lato}

\usepackage{unicode-math}
\mathitalicsmode=1

\setmathfont[
mathit = sym,
mathup = sym,
mathbf = sym,
math-style = TeX, 
bold-style = TeX
]{Wholegrain Math}
\everydisplay{\Umathoperatorsize\displaystyle=4ex}
\AtBeginDocument{\renewcommand\setminus{\smallsetminus}}
% font setup ends here

\newcommand\extrafootertext[1]{%
    \bgroup
    \renewcommand\thefootnote{\fnsymbol{footnote}}%
    \renewcommand\thempfootnote{\fnsymbol{mpfootnote}}%
    \footnotetext[0]{#1}%
    \egroup
}

\theoremstyle{definition}
\newtheorem{problem}{Problem}
\newenvironment{solution}
  {\textbf{Solution.} }
  {}

\usepackage{listings}
\lstset{basicstyle=\ttfamily,breaklines=true}

\usepackage{enumitem}
\setlist[itemize]{itemsep=-.15em, topsep=-.5em}
\setlist[enumerate]{itemsep=-.15em, topsep=-.5em}
\setlist[enumerate,1]{label={(\alph*)}}

\PassOptionsToPackage{hyphens}{url}
\usepackage{hyperref}

\setlength{\parskip}{1em}
\setlength{\parindent}{0em}

\renewcommand{\d}{\mathrm{d}}
\renewcommand{\vec}{\mathbf}
\newcommand{\T}{\mathsf{T}}
\newcommand{\F}{\mathsf{F}}

\usepackage{dsfont}
\newcommand{\bOne}{\mathds{1}}

\usepackage{array,booktabs}

\begin{document}

\textbf{\Large\sffamily{Homework 3\hfill Numerical Analysis Fall 2024}}
    
    \vspace{-1.8em}
    \hrulefill
 
\textbf{Instructions:}
    \begin{itemize}
        \item Due 10/7 at 6:00pm on \href{https://www.gradescope.com/courses/818054}{Gradescope}.
        \item You must follow the submission policy in the \href{https://courses.chen.pw/na_f2024/syllabus.html}{syllabus} 
\end{itemize}
   
\vspace{.5em}


\begin{problem}
    For each problem, write down a matrix or matrices which performs the stated operations to a $3\times 4$ matrix.
    Make sure to specify whether you should be applied on the left or right.

    For reference, we will also show what the operation does to the following matrices:
    \[
        \vec{A} = 
        \begin{bmatrix}
            1 & 2 & 3 & 4 \\
            5 & 6 & 7 & 8 \\
            9 & 10 & 11 & 12
        \end{bmatrix}
        ,\qquad 
        \vec{B} = 
        \begin{bmatrix}
            1 & 3& -2 & 0 \\
            2 & -3 & 1 & 1 \\
            2 & -2 & 5 & 4
        \end{bmatrix}
    \]

    \begin{enumerate}
        \item Extract the second column.
            \[
                \vec{A} \to \begin{bmatrix} 2 \\ 6 \\ 10 \end{bmatrix}
                    ,\qquad
                    \vec{B} \to \begin{bmatrix} 3 \\ -3 \\ -2 \end{bmatrix}
            \]
        \item Extract the second column and place it in the third column of a $3\times 3$ matrix of zeros.
            \[
                    \vec{A} \to \begin{bmatrix} 0&0&2 \\ 0&0&6 \\ 0&0&10 \end{bmatrix}
                    ,\qquad
                    \vec{B} \to \begin{bmatrix} 0&0&3 \\ 0&0&-3 \\ 0&0&-2 \end{bmatrix}
            \]
        \item Swap the second and third columns.
    \[
        \vec{A} \to 
        \begin{bmatrix}
            1 & 3 & 2 & 4 \\
            5 & 7 & 6 & 8 \\
            9 & 11 & 10 & 12
        \end{bmatrix}
        ,\qquad 
        \vec{B} \to
        \begin{bmatrix}
            1 & -2 & 3& 0 \\
            2 & 1 & -3 & 1 \\
            2 & 5 & -2 & 4
        \end{bmatrix}
    \]
    \item Sum up each column.
    \[
        \vec{A} \to  
        \begin{bmatrix}
            15 & 18 & 21 & 24 
        \end{bmatrix}
        ,\qquad 
        \vec{B} \to 
        \begin{bmatrix}
            5 & -2 & 4 & 5
        \end{bmatrix}
    \]
    \item Swap the second and third columns, then sum up each column  (this one requires using two matrices).
    \[
        \vec{A} \to  
        \begin{bmatrix}
            15 & 21 & 18 & 24 
        \end{bmatrix}
        ,\qquad 
        \vec{B} \to 
        \begin{bmatrix}
            5 & 4 & -2 & 5
        \end{bmatrix}
    \]

    \item Sum all the entries (this one requires using two matrices).
    \[
        \vec{A} \to  
        \begin{bmatrix}
            78
        \end{bmatrix}
        ,\qquad 
        \vec{B} \to 
        \begin{bmatrix}
            12
        \end{bmatrix}
    \]

    \end{enumerate}
\end{problem}


\begin{problem}
    Plotting data is often a useful way to understand the behavior of a function. For instance, we might plot the runtime of an algorithm as a function of the size of the input.
    The simplest thing to do is plot $x$ vs $y$ values. 
    However, sometimes it is more useful to plot $x$ vs $\log(y)$ or $\log(x)$ vs $\log(y)$.
    This problem will explore why this is the case.

    \begin{enumerate}
        \item Prove that on a log-log plot, $n$ vs $c n^k$ is a line. What is the slope? What is the intercept?
        \item What will the plot of $n$ vs $50n^2 + 3n + 1$ look like on a log-log plot when $n$ is large?
        \item Prove that on a log-y plot, $n$ vs $\rho^n$ is a line. What is the slope?
        \item Over the range $n\in[1,20]$ make a plot of:
        \begin{itemize}
            \item  $n$ vs $50n^2+3n+1$ and $n$ vs $n^3$ 
            \item $n$ vs $\log(50n^2+3n+1)$ and $n$ vs $\log(n^3)$
            \item $\log(n)$ vs $\log(50n^2+3n+1)$ and $\log(n)$ vs $\log(n^3)$
        \end{itemize}
        
        Imagine you didn't know what functions we were plotting (i.e. you were timing two algorithms and got the plots empirically). 
        Explain why the log-log plot is the most useful for understanding the behavior of growth of the functions beyond $n=20$.
    \end{enumerate}

\end{problem}



\begin{problem}

    In chapter 9, we see our implementation of a triangular solved was faster than numpy's \lstinline{np.linalg.solve}.

    Scipy has a solver \href{https://docs.scipy.org/doc/scipy/reference/generated/scipy.linalg.solve_triangular.html}{\lstinline{sp.linalg.solve_triangular}} for triangular systems which uses LAPACK's triangular solver.

    Add this solver to the runtime comparison of \lstinline{np.linalg.solve} and our triangular solver in Chapter 9 of the notes (all the code from the notes is online on the Google Drive).
    
    Include the new figure showing the runtime scaling of all three algorithms.
    How does scipy's triangular solver compare?

    Remember to include your code as well as the figure.

\end{problem}

\begin{problem}
In lecture, we said floating point numbers were basically like scientific notation.

Let's define the set of numbers which we can represent in finite precision with $p>0$ decimal places as 
\[
\mathbb{F}_p := \big\{ \pm b.b_1b_2\ldots b_p \times 10^{m} : b\in\{1,\dots, 9\}, b_i\in \{0,\ldots, 9\}, m\in\mathbb{Z} \big\} 
\]
For example, $3.2414\times 10^2$ and $9.9999 \times 10^{-99}$ are both in $\mathbb{F}_4$.


Define the rounding function $\operatorname{round} : \mathbb{R} \to \mathbb{F}_p$ as the function which takes in a number $x$ and outputs the nearest number $\operatorname{round}(x)$ in $\mathbb{F}_p$ to $x$ (round up in a tie).

For example, 
\[
x = \pi = 3.14159265358979323846\ldots \to \operatorname{round}(x) = 3.1416 \times 10^0.
\]

\begin{enumerate}
    \item What is the distance between two consecutive numbers in $\mathbb{F}_p$ with the same exponent $m$?
    \item Prove that for any $x\in\mathbb{R}$, 
    \[ 
        |x - \operatorname{round}(x) | \leq 10^{-p} |x|.
    \]
    \emph{Hint}: Use (a) and argue that $|x| \geq 10^{m}$. 
    
    \item Note that $\mathbb{F}_p$ is an infinite set. Define 
    \[
    \mathbb{F}_{p,q} := \big\{ \pm b.b_1b_2\ldots b_p \times 10^{m} : b\in\{1,\dots, 9\}, b_i\in \{0,\ldots, 9\}, m\in\{-q,-q+1 \ldots, q\} \big\}.
    \]
    This is like $\mathbb{F}_p$ but we only allow exponents between $-q$ and $q$.

    Show that if $\operatorname{round} : \mathbb{R} \to \mathbb{F}_{p,q}$ is the function which takes in a number $x$ and outputs the nearest number $\operatorname{round}(x)$ in $\mathbb{F}_{p,q}$ to $x$ (round up in a tie) there exits $x\in\mathbb{R}$ with
    \[ 
        |x - \operatorname{round}(x) | > 10^{-p} |x|.
    \]
    
    
\end{enumerate}

\end{problem}

\begin{problem}

    Define,
    \begin{equation*}
        \vec{A} =
        \begin{bmatrix}
         3 & 4 & 1 & -8 \\
         6 & 5 & 2 & -28 \\
         3 & 2 & -8 & -13 \\
         -12 & -15 & -4 & 48 \\
        \end{bmatrix}
        ,\qquad
        \vec{b} = 
        \begin{bmatrix}-10\\-26\\-38\\54\end{bmatrix}
        \end{equation*}
        We want to find $\vec{x}$ so that $\vec{A}\vec{x} = \vec{b}$.
        Suppose we have
        \[
            \vec{A} = \vec{L}\vec{U},\qquad
        \vec{L} = 
        \begin{bmatrix}
     1 & 0 & 0 & 0 \\
     2 & -3 & 0 & 0 \\
     1 & -2 & 3 & 0 \\
     -4 & 1 & 0 & 4 \\
        \end{bmatrix}
        ,\qquad
        \vec{U} = 
        \begin{bmatrix}
     3 & 4 & 1 & -8 \\
     0 & 1 & 0 & 4 \\
     0 & 0 & -3 & 1 \\
     0 & 0 & 0 & 3 \\
        \end{bmatrix}
        \]
        We will use this factorization to solve $\vec{A} \vec{x} = \vec{b}$.

        You should show all the steps of computing this quantity by hand (but can use a calculator for arithmetic). You should make sure you understand how to do this by hand as this type of problem would be fair game for a quiz.

        \begin{enumerate}
        \item Solve $\vec{L}\vec{y} = \vec{b}$.
        \item Solve $\vec{U}\vec{x} = \vec{y}$.
        \item Check that $\vec{A}\vec{x} = \vec{b}$.
        \end{enumerate}
    
    \end{problem}
    
    \clearpage
    \begin{problem}
    Define,
    \begin{equation*}
        \vec{A} =
        \begin{bmatrix}
     1 & 5 & 7 & 8 \\
     1 & -1 & 1 & 6 \\
     1 & 5 & 5 & -2 \\
     1 & -1 & -1 & 4 \\
        \end{bmatrix}
        ,\qquad
        \vec{b} = 
        \begin{bmatrix}-18\\-90\\-69\\138\end{bmatrix}
        \end{equation*}
        We want to find $\vec{x}$ so that $\vec{A}\vec{x} = \vec{b}$.
        Suppose we have
        \[
        \vec{A} = \vec{Q}\vec{R},\qquad
        \vec{Q} = 
        \frac{1}{2}\begin{bmatrix}
     1 & 1 & 1 & 1 \\
     1 & -1 & 1 & -1 \\
     1 & 1 & -1 & -1 \\
     1 & -1 & -1 & 1 \\
        \end{bmatrix}
        ,\qquad
        \vec{R} = 
        \begin{bmatrix}
     2 & 4 & 6 & 8 \\
     0 & 6 & 6 & -2 \\
     0 & 0 & 2 & 6 \\
     0 & 0 & 0 & 4 \\
        \end{bmatrix}
        \]
        We will use this factorization to solve $\vec{A} \vec{x} = \vec{b}$.

        You should show all the steps of computing this quantity by hand (but can use a calculator for arithmetic). You should make sure you understand how to do this by hand as this type of problem would be fair game for a quiz.

        \begin{enumerate}
            \item 
            What is $\vec{Q}^\T \vec{Q}$? What does this tell you about $\vec{Q}^{-1}$?
            \item Solve $\vec{Q}\vec{y} = \vec{b}$.
            \item Solve $\vec{R}\vec{x} = \vec{y}$.
            \item Check that $\vec{A}\vec{x} = \vec{b}$.
        \end{enumerate}

    \end{problem}

\begin{problem}
Describe whether the following problems/tasks are well-conditioned or not. 
    Justify each of your responses.

    \begin{enumerate}
         \item Problem/task: You are given a vector $[x_1, x_2]$ and must compute the solution $[z_1, z_2]$ to the linear system of equations
            \begin{equation*}
                 2.3z_1 − 1.01z_2 = x_1 
                 ,\qquad 
                 2.31 z_1 − 1.00 z_2= x_2.
            \end{equation*}
            
            Example inputs/outputs:

            \begin{center}
            \begin{tabular}{>{\centering\arraybackslash}m{2in}>{\centering\arraybackslash}m{2in}}
            \toprule
                input $x$ & solution $P(x)$ \\ \midrule
                $[1,2]$ & $[30.816,69.184]$ \\
                $[1,1]$ & $[0.30211,-0.302115]$ \\ 
                \bottomrule
            \end{tabular}
            \end{center}

        \item Problem/task: You are given function $h:[-1,1]\to \mathbb{R}$ and must return $\int_{-1}^{1} h(s) \d{s}$. 
            We will use the norm $\|\tilde{h} - h\|_\infty := \max_{s\in[-1,1]} |\tilde{h}(s) - h(s)|$.

            Example inputs/outputs:

            \begin{center}
            \begin{tabular}{>{\centering\arraybackslash}m{2in}>{\centering\arraybackslash}m{2in}}
            \toprule
                input $x$ & solution $P(x)$ \\ \midrule
                $h(s) = 1$ & 2 \\
                $h(s) = s^2$ & 2/3 \\
                $h(s) = \sin(s)$ & 0 \\
                \bottomrule
            \end{tabular}
            \end{center}

\clearpage

\item Problem/task: You're designing a self-driving car. 
            As you approach an intersection, you are given a greyscale image of the stoplight. 
            You must determine whether it is green or red; i.e. if the car should continue through the intersection or stop.
        
            Example inputs/outputs:

            \begin{center}
            \begin{tabular}{>{\centering\arraybackslash}m{2in}>{\centering\arraybackslash}m{2in}}
            \toprule
                input $x$ & solution $P(x)$ \\ \midrule
            \includegraphics[width=1.75in]{green_day_gs.jpg} & green \\
            \includegraphics[width=1.75in]{red_day_gs.jpg} & red \\
            \includegraphics[width=1.75in]{green_night_gs.jpg} & green \\
            \includegraphics[width=1.75in]{red_night_gs.jpg} & red \\
                \bottomrule
            \end{tabular}
            \end{center}
\clearpage


        \item Problem/task: You're designing a self-driving car. 
            As you approach an intersection, you are given a color image of the stoplight. 
            You must determine whether it is green or red; i.e. if the car should continue through the intersection or stop.
            
            Example inputs/outputs:

            \begin{center}
            \begin{tabular}{>{\centering\arraybackslash}m{2in}>{\centering\arraybackslash}m{2in}}
            \toprule
                input $x$ & solution $P(x)$ \\ \midrule
            \includegraphics[width=1.75in]{green_day.jpg} & green \\
            \includegraphics[width=1.75in]{red_day.jpg} & red \\
            \includegraphics[width=1.75in]{green_night.jpg} & green \\
            \includegraphics[width=1.75in]{red_night.jpg} & red \\
                \bottomrule
            \end{tabular}
            \end{center}

       
            Images from:
            \href{https://www.youtube.com/watch?v=yhCAWZlv_Sc}{Brooklyn 4K - Night Drive}
            and \href{https://www.youtube.com/watch?v=eyu0d1x5TMs}{Jazz New York City Drive}

    \end{enumerate}
\end{problem}

\end{document}
